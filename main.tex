\documentclass[11pt]{article}
\usepackage{times}
\usepackage{setspace}
\usepackage{geometry}
\geometry{a4paper,left=1in,right=1in,top=1in,bottom=1in}
\usepackage{amsmath}
\usepackage[round, authoryear]{natbib}
\usepackage{xcolor}
\usepackage{pdflscape} % for landscape pages
\usepackage{longtable} % for multipage tables
\usepackage{array}     % for better column control
\usepackage[justification=raggedright,singlelinecheck=false]{caption}

\begin{document}
\begin{landscape} 
    \begin{longtable}{>{\raggedright\arraybackslash}p{2cm} 
        >{\raggedright\arraybackslash}p{4cm} 
        >{\raggedright\arraybackslash}p{10cm} 
        >{\raggedright\arraybackslash}p{6cm}}
\caption{Summary of machine learning literature: predicted quantities, features, and training data} \\
\hline
Author (Year) & Quantity Predicted & Feature Variables & Training Data \\
\hline
\endfirsthead

\multicolumn{4}{l}%
{{\bfseries Table \thetable{} continued}} \\
\hline
Author (Year) & Quantity Predicted & Feature Variables & Training Data \\
\hline
\endhead

\hline
\multicolumn{4}{r}{{Continued on next page}} \\
\endfoot

\hline
\endlastfoot

\citet{reid2015spatiotemporal} & Daily PM$_{2.5}$ concentrations during the 2008 Northern California wildfires (June~20 – July~31, 2008) & GOES-West Geostationary Operational Environmental Satellite Aerosol/Smoke Product AOD at 4~km resolution every 30~min during daylight, MODIS AOD at 10~km twice daily, local AOD at 0.5~km resolution from Sonoma Technology, WRF-Chem chemical transport model PM$_{2.5}$ estimates at 12~km resolution, meteorological variables (temperature, relative humidity, sea level and surface pressure, planetary boundary layer height, dew point, wind components) from Rapid Update Cycle (RUC) model, distance to nearest fire cluster and fire counts, traffic and land use variables, elevation, spatial coordinates, temporal variables (Julian date, weekend indicator) & Ground-level PM$_{2.5}$ measurements from 112 monitoring stations including Federal Reference and Equivalent Method monitors in Northern California during the wildfire period \\[6pt]

\citet{chen2018machine} & Daily PM$_{2.5}$ concentrations across China (2005–2016) & MODIS satellite-retrieved Aerosol Optical Depth (AOD) data (2005–2016), AOD merged from Deep Blue and Dark Target products,Daily meteorological variables (Temperature, Relative Humidity, Barometric Pressure, Wind Speed) from 824 weather stations, Monthly normalized difference vegetation index (NDVI), Urban land cover (500~m resolution), Day of year, Elevation (log-transformed) & Ground-level PM$_{2.5}$ measurements from 1479 monitoring stations across China (2014–2016) \\[6pt]

\citet{xiao2018ensemble} & Daily ground-level PM$_{2.5}$ concentrations across China (2013–2016 for training, 2017 for hindcast evaluation) & MODIS Terra and Aqua Collection 6 Deep Blue and Dark Target Aerosol Optical Depth (AOD) products at 10~km resolution (gap-filled by multiple imputation), meteorological variables including temperature, humidity, precipitation, planetary boundary layer height, OMI tropospheric NO$_2$, aerosol absorption index (AAI), visibility, MERRA-2 PM$_{2.5}$ simulations, population density, elevation, fire counts within buffer zones, and NDVI & Hourly ground-level PM$_{2.5}$ measurements from \(\sim\)1593 air quality monitoring stations across mainland China, Hong Kong, and Taiwan (2013–2017) \\[6pt]

\citet{xu2018evaluation} & Monthly ground-level PM$_{2.5}$ concentrations across British Columbia, Canada (2001–2014) & MODIS Terra MOD04\_3K 3~km daily AOD, MODIS Terra MOD11A1 1~km daily Land Surface Temperature (LST), MODIS Terra MOD05\_L2 1~km daily water vapor product, MODIS Terra MOD13A3 1-km monthly NDVI, MODIS Terra/Aqua MCD43B3 1~km 8~day albedo, NCAR/NCEP reanalysis monthly planetary boundary layer height and wind speed, Shuttle Radar Topography Mission (SRTM) 90~m elevation, distance to ocean, calendar month & Ground-level PM$_{2.5}$ measurements from 63 National Air Pollution Surveillance (NAPS) stations with hourly data aggregated monthly (2001–2014) \\[6pt]

\citet{chen2019stacking} & Hourly PM$_{2.5}$ concentrations in Central and Eastern China (2016) & Himawari-8 geostationary satellite hourly Aerosol Optical Depth (AOD) data at 0.18° spatial resolution,
Meteorological variables from ECMWF ERA-Interim reanalysis (Relative Humidity, Boundary Layer Height, Wind speed, Surface Pressure, Temperature), Normalized Difference Vegetation Index (NDVI) from MODIS 16~day composite, Digital Elevation Model at 90~m resolution & Ground-level PM$_{2.5}$ hourly measurements from China Environmental Monitoring Centre (CEMC) in 2016 \\[6pt]

\citet{karimian2019evaluation} & Hourly PM$_{2.5}$ concentration forecasts in Tehran, Iran (2013–2016) & Hourly meteorological variables (temperature, planetary boundary layer height, surface pressure, wind components, relative humidity) from ECMWF reanalysis data at 0.1° resolution, Ground-level PM$_{2.5}$ hourly measurements from 9 monitoring stations in Tehran (2013–2016), Temporal features including forecast horizon & Multiple additive regression trees, deep feed-forward neural network, and long short-term memory neural network models trained and evaluated on historical and forecasted meteorological and PM$_{2.5}$ data \\[6pt]

\citet{li2019predicting} & Daily ground-level PM$_{2.5}$ concentrations in the Beijing-Tianjin-Hebei (BTH) region, China (2015–2017) & MODIS Aqua satellite Deep Blue Collection 6.1 (C6.1) Aerosol Optical Depth (AOD) products at 10 km resolution, meteorological variables including air temperature, relative humidity, wind speed and direction, pressure from NCEP/NCAR reanalysis, air pollutant concentrations (SO$_2$, NO$_2$, CO, O$_3$) from ground monitoring, geographical variables and temporal features & Hourly ground-level PM$_{2.5}$ measurements from 78 monitoring stations across BTH region (2015–2017) \\[6pt]

\citet{chen2020estimating} & Daily PM$_{2.5}$ concentrations over Shenzhen, China (2016–2018) & MODIS Terra and Aqua MAIAC 1~km Aerosol Optical Depth (AOD) products (2016–2018), Daily meteorological parameters including Relative Humidity (RH) and Extreme Wind Speed (EWS) interpolated at 1~km resolution, Geographical location (latitude, longitude), temporal variables (day/time), elevation & 
Hourly ground-level PM$_{2.5}$ measurements from 19 monitoring sites in Shenzhen provided by China Environmental Monitoring Center (2016–2018) \\[6pt]

\citet{gupta2021machine} & Hourly and daily surface PM$_{2.5}$ concentrations in Thailand (2018) & NASA MERRA-2 aerosol reanalysis data (aerosol components including dust, sea salt, black carbon, organic carbon, sulfate), meteorological parameters (temperature, relative humidity, planetary boundary layer height, wind speed, pressure), satellite-derived Aerosol Optical Depth (AOD), geographic (latitude, longitude) and temporal variables & Ground-level PM$_{2.5}$ hourly and daily measurements from 51 monitoring stations across Thailand for year 2018 \\[6pt]

\citet{chen2022obtaining} & Vertical distribution of PM$_{2.5}$ in five regions of China (2015–2019) & CALIOP Level-2 Aerosol Profiles (APRO) Version 4.20 at 532~nm with 60~m vertical and 5~km horizontal resolution (2015–2019), Meteorological factors including ozone concentration, air pressure, temperature, relative humidity, wind speed from MERRA-2 and ERA-5 reanalysis datasets, Ground-level hourly PM$_{2.5}$ measurements from 729 stations in Beijing-Tianjin-Hebei, Central China, Cheng-Yu, Pearl River Delta, and Yangtze River Delta regions & Ground-level PM$_{2.5}$ hourly data from China Environmental Monitoring Center (CEMC) (2015–2019) \\[6pt]

\citet{just2020advancing} & Daily PM$_{2.5}$ concentrations at 1~km$\times$1~km resolution over Northeastern USA (2000–2015) & MAIAC aerosol optical depth (AOD) retrievals from MODIS Aqua and Terra satellites at 1~km resolution, United States EPA Air Quality System (AQS) ground-based PM$_{2.5}$ measurements, planetary boundary layer height (PBL), inverse distance weighted (IDW) surface PM$_{2.5}$ estimates, percentage of developed area from USGS National Land Cover Dataset (NLCD), meteorological variables including temperature, relative humidity, and others & Ground-level PM$_{2.5}$ daily measurements from 388 monitoring sites across 13 states and DC (2000–2015) from EPA AQS database \\[6pt]

\citet{karimian2023evaluation} & Hourly PM$_{2.5}$ concentrations in the Yangtze River Delta region, China (2016–2017) & Himawari-8 geostationary satellite Level-3 Aerosol Optical Depth (AOD) at 0.05° spatial resolution and hourly temporal resolution, meteorological variables from ECMWF reanalysis including 2-m temperature, boundary layer height, relative humidity, 10-m wind components, surface pressure, forecasted albedo, and ground-level pollutants (PM$_{10}$, SO$_2$, NO$_2$, CO, O$_3$), Wavelet transform decomposed PM$_{2.5}$ time series features, and feature selection via minimum redundancy maximum relevance (mRMR) & Hourly ground-level PM$_{2.5}$ measurements from 137 monitoring sites in the Yangtze River Delta region during 2016–2017 \\[6pt]

\citet{kulkarni2022model} & Daily mean PM$_{2.5}$ concentrations in National Capital Territory (NCT), Delhi, India (2019) & MODIS-MAIAC combined Terra and Aqua satellite AOD data at 1~km resolution, ERA5-Land reanalysis meteorological parameters including 2~m temperature, relative humidity, wind speed and direction, surface pressure, planetary boundary layer height, MODIS NDVI and Column Water Vapor (CWV), ETOPO2 elevation data & Hourly ground-level PM$_{2.5}$ measurements from 38 Continuous Ambient Air Quality Monitoring Stations (CAAQMS) across NCT Delhi during 2019 \\[6pt]

\citet{lee2021air} & Hourly ground-level PM$_{10}$ and PM$_{2.5}$ concentrations over East Asia (April–May 2019) & Geostationary Ocean Color Imager (GOCI) satellite Aerosol Optical Depth (AOD) data at 6~km$\times$6~km spatial resolution with hourly temporal coverage (00–07~UTC), meteorological variables and surface PM observations from Korea, China, and Japan, Random Forest (RF) machine learning estimates of PM from AOD, and Weather Research and Forecasting Chemistry (WRF-Chem) numerical model data & Hourly ground-based PM$_{10}$ and PM$_{2.5}$ measurements from AirKorea (South Korea), Beijing Municipal Environmental Protection Monitoring Center (China), and National Institute for Environmental Studies (Japan) during April–May 2019 \\[6pt]

\citet{liu2019satellite} & Hourly surface PM$_{2.5}$ concentrations across mainland China (2016) & Advanced Himawari Imager (AHI) satellite top-of-atmosphere (TOA) reflectances at 0.47~\(\mu \rm m\), 0.64~\(\mu \rm m\), and 2.3~\(\mu \rm m\) wavelengths, observation angles (solar and satellite zenith and azimuth angles), meteorological variables from ERA-Interim reanalysis including surface pressure, total column water, 10-m wind components, 2-m air temperature, total column ozone, relative humidity, planetary boundary layer height, spatial-temporal variables (latitude, longitude, month, day, hour), and NDVI-like vegetation indices & Hourly ground-level PM$_{2.5}$ measurements from $\sim$1500 monitoring sites across China for year 2016 \\[6pt]

\citet{minh2021pm2} & Hourly PM$_{2.5}$ concentration forecasts for Ho Chi Minh City, Vietnam (2016–2020) & Meteorological data simulated by WRF model (temperature, humidity, wind speed and direction), historical observed PM$_{2.5}$ and meteorological data from Tan Son Nhat station, and other meteorological variables from NOAA NCEI & Ground-level hourly PM$_{2.5}$ measurements from U.S. Consulate monitoring station in Ho Chi Minh City (2016–2020) \\[6pt]

\citet{stirnberg2020mapping} & Hourly PM$_{10}$ concentrations in Germany (2007–2015) & MODIS MAIAC satellite Aerosol Optical Depth (AOD) at 1~km resolution, MODIS Normalized Difference Vegetation Index (NDVI), NASA VIIRS Nighttime Lights data, EU-DEM elevation and topographic position index (TPI), CORINE Land Cover classifications, ERA-Interim reanalysis meteorological variables including air pressure, relative humidity, temperature, wind components, boundary layer height (BLH), precipitation data from RADOLAN, EEA emission data, day of year (DOY), and day of week & Hourly ground-level PM$_{10}$ measurements from the German Federal Environmental Agency (UBA) stations (2007–2015) \\[6pt]

\citet{liang2022prediction} & Cloud condensation nuclei (CCN) spectral parameters in the North China Plain during four field campaigns (2016–2019) & Aerosol optical properties including scattering coefficient ($\sigma_{\rm sp}$), backscattering coefficient ($\sigma_{\rm bsp}$), hemispheric backscattering fraction (HBF), single scattering albedo (SSA), Ångström exponent, mass concentrations of black carbon (mBC), nitrate (NO$_3^-$), sulfate (SO$_4^{2-}$), chloride (Cl$^-$), ammonium (NH$_4^+$), and organic compounds & Observational aerosol and CCN spectral data from four intensive campaigns in the North China Plain during autumn and winter seasons (2016, 2018, 2019) \\[6pt]

\citet{nair2020using} & CCN at 0.4\% supersaturation globally and at the ARM Southern Great Plains site, USA (1989–2018 simulations, 2007–2020 observations) & Atmospheric state and composition predictors including eight PM$_{2.5}$ fractions (NH$_4^+$, SO$_4^{2-}$, NO$_3^-$, secondary organic aerosol (SOA), black carbon (BC), primary organic carbon (POC), dust, salt), seven gaseous species (NO$_x$, NH$_3$, O$_3$, SO$_2$, OH, isoprene, monoterpene), and four meteorological variables (temperature, relative humidity, precipitation, solar radiation) & 30-year GEOS-Chem chemical transport model with advanced particle microphysics simulations (1989–2018) for training; Observational measurements at ARM SGP site for validation \\[6pt]

\citet{nair2021machine} & CCN at 0.4\% supersaturation across multiple airborne campaigns covering varied tropospheric physicochemical regimes (various years) & Atmospheric state and composition predictors including temperature (T), relative humidity (RH), gas-phase species ([SO$_2$], [NO$_x$], [O$_3$]), and aerosol mass speciation (NH$_4^+$, SO$_4^{2-}$, NO$_3^-$, organic aerosol (OA)) from airborne measurements, combined with meteorological variables & Multi-campaign airborne in situ measurements from seven field campaigns (ARCTAS, ATom1-4, DC3, DISCOVER-AQTX, KORUS-AQ, SEAC4RS, WE-CAN) spanning different seasons and tropospheric heights \\[6pt]

\citet{yu2022use} & Global particle number concentration for particles $>$10~nm diameter (1989–2018 simulations) & Outputs from the GEOS-Chem size-resolved aerosol microphysics model (GCAPM) including mass concentrations of sulfate, nitrate, ammonium, secondary organic aerosol, black carbon, primary organic carbon, sea salt, dust; gaseous species concentrations (SO$_2$, NH$_3$, NO$_x$, O$_3$, OH, isoprene, monoterpenes); meteorological variables including temperature, relative humidity, and pressure & Long-term (30-year) GCAPM simulation outputs at 47 global sites; validation with observational PNC measurements at 35 sites worldwide \\[6pt]

\citet{park2023new} & CCN at 0.6\% supersaturation over the Korean Peninsula (2017, 2019, 2021, 2006–2010, 2018–2020) & Aerosol number size distribution data measured by SMPS and CCN counters onboard a research vessel over the Yellow Sea and at Yonsei University campus in Seoul; data preprocessing includes non-negative matrix factorization for dimensionality reduction and multiple linear regression to predict N$_{\rm CCN}$ & Independent training and test data sets separated by years, covering maritime (Yellow Sea) and urban (Seoul) environments with simultaneous aerosol size distribution and N$_{\rm CCN}$ measurements \\[6pt]

\citet{rejano2024ccn} & CCN at a high-altitude remote site in Sierra Nevada, Spain during summer 2021 & Aerosol chemical composition including organic aerosol (OA) factors derived from positive matrix factorization such as hydrocarbon-like OA (HOA), less-oxidized oxygenated OA (LO-OOA), more-oxidized oxygenated OA (MO-OOA); aerosol inorganic species (SO$_4^{2-}$, NO$_3^-$, NH$_4^+$, Cl$^-$); equivalent black carbon (eBC); aerosol particle number size distribution (PNSD); solar global irradiance; neural network inputs include CCN number concentration for particles $>$80~nm (N$_{80}$), OA fraction, $f_{\rm 44}$ oxidation parameter, and solar irradiance & Observations from BioCloud intensive summer field campaign at Sierra Nevada station, Spain (June–July 2021) including ToF-ACSM aerosol chemical speciation, CCN counter measurements at multiple supersaturations, SMPS particle size distribution, and meteorological measurements \\[6pt]

\citet{annapurna2024composition} & Aerosol classification by composition and source based on AERONET data from Kanpur, India (2002–2018) & Aerosol optical properties including Aerosol Optical Depth (AOD at 500~nm), Angstrom Exponent (AE at 440–870~nm), Single Scattering Albedo (SSA at 440~nm), Fine Mode Fraction (FMF at 550 nm) & AERONET level 2.0 inversion products with hourly averaged data from Kanpur site (2002–2018) \\[6pt]

\citet{choi2021first} & Aerosol type classification into seven classes (pure dust, dust-dominant mixed, pollution-dominant mixed, strongly, moderately, weakly absorbing pollution, and non-absorbing sulfate aerosols) globally using satellite data from 2018 to 2020 & Satellite variables including aerosol index and trace gas columns (CO, NO$_2$) from TROPOMI, aerosol optical depth (AOD), Ångström exponent (AE), top-of-atmosphere (TOA) reflectance at 412, 470, and 660~nm, land cover type, urban area fraction, and solar zenith angle from MODIS and TROPOMI sensors & AERONET Level 1.5 and 2.0 inversion product aerosol type dataset (based on particle linear depolarization ratio (PLDR) and single scattering albedo (SSA)) for training and validation (January 2018 – July 2020) \\[6pt]

\citet{christopoulos2018machine} & Aerosol classification of $\sim$20 chemically distinct aerosol types and four broad aerosol categories using SPMS spectra from laboratory and field experiments & 512 mass spectral features from SPMS (positive and negative ion modes), aerodynamic particle size; feature selection applied to reduce dimensionality & Training data from $\sim$50,000 SMPS collected at KIT AIDA chamber and MIT Aerosol and Cloud Laboratory during intercomparison experiments (FIN01) \\[6pt]

\citet{heikkinen2021eight} & Eight years of sub-micrometre organic aerosol (OA) composition data at the boreal forest SMEAR II station in Finland (2012–2019) & Aerosol Chemical Speciation Monitor (ACSM) mass spectra of non-refractory submicron particulate matter (PM$_{1}$) with 3~hour averaged OA mass concentrations; wind and meteorological data; positive matrix factorization results processed with k-means clustering and rolling relaxed chemical mass balance (rCMB) approaches & Long-term ACSM measurements at SMEAR II station (61.85°N, 24.29°E) in southern Finland from April 2012 to September 2019 \\[6pt]

\citet{li2022aerosol} & Global aerosol regimes classification in three atmospheric layers (lower troposphere, middle troposphere, tropopause) based on multi-year means from 2000 to 2013 simulations & Seven aerosol parameters from the EMAC-MADE3 global chemistry–climate model: mass concentrations of mineral dust, black carbon, particulate organic matter (POM), sea salt, sulfate+nitrate+ammonium (SNA), and particle number concentrations of Aitken and accumulation modes & Global EMAC-MADE3 simulations (2000–2013) with 1.9°$\times$1.9° horizontal resolution and 31 vertical levels, averaged into three atmospheric layers \\[6pt]

\citet{ningombam2024aerosol} & Global aerosol classification into four main types (dust, biomass burning, urban/industrial, mixed aerosols) based on 39,411 daily observations at 150 AERONET sites across six continents (1993–2022) & Aerosol optical parameters including fine-mode aerosol optical depth, extinction Ångström exponent, absorption Ångström exponent, single scattering albedo, and real refractive index at 440~nm from AERONET Version 3.0 Level 2.0 inversion products; land use land cover data for site classification & Quality-controlled AERONET aerosol inversion data from 150 global sites spanning Africa, Asia, Australia, Europe, North and South America (1993–2022) \\[6pt]

\citet{siomos2020automated} & Aerosol classification into Fine Non Absorbing, Black Carbon mixtures, Dust mixtures, and Mixed types over Thessaloniki, Greece (1998–2017) & Spectral UV measurements from a double monochromator Brewer spectrophotometer including Single Scattering Albedo (SSA) at 340~nm and Extinction Angstrom Exponent (EAE) at 320–360 nm; training supported by coincident CIMEL sunphotometer data (Fine Mode Fraction and SSA at 440~nm) & Brewer spectrophotometer measurements (1998–2017) at Thessaloniki, Greece; coincident CIMEL sunphotometer data (2003–2017) for training and validation \\[6pt]

\citet{song2022understanding} & Size-resolved aerosol particle number concentrations (10~nm to 20~\(\mu \rm m\)) at the Gruvebadet Observatory, Svalbard, Arctic during March to October 2015 & Aerosol size distributions measured by Scanning Mobility Particle Sizer (SMPS, 10.4–469.8~nm) and Aerodynamic Particle Sizer (APS, 0.542–19.81~\(\mu \rm m\)); daily chemical composition data from filter samples including ions (Cl$^-$, Br$^-$, NO$_3^-$, SO$_4^{2-}$, NH$_4^+$, K$^+$, Mg$^{2+}$, Ca$^{2+}$), organic anions (methanesulfonate, oxalate), metals (Al, As, Ba, Cd, Ce, Cr, Cu, Fe, La, Mn, Ni, Pb, Ti, V, Zn); meteorological variables including wind speed, direction, temperature, relative humidity, surface pressure, cloud cover, snowfall, solar radiation, and boundary layer height; air mass back trajectories and concentration weighted trajectories (CWT) & Measurements and sampling at Gruvebadet Observatory, Svalbard (78.9°N, 11.9°E), March–October 2015 \\[6pt]

\citet{wang2022machine} & Concentrations of volatile organic compounds (VOCs), ozone (O$_3$), and secondary organic aerosol (SOA) in Tianjin, China (June 2017 – May 2019) & Fifty-four VOC species measured hourly during daytime (8:00–19:00), ozone (O$_3$) concentrations, PM$_{2.5}$ organic carbon (OC), elemental carbon (EC); machine learning inputs include subsets of VOCs (alkenes, alkanes, aromatics) & Continuous ground-based measurements at Nankai University Air Quality Research Supersite (NKAQRS), Tianjin, China from June 2017 to May 2019 \\[6pt]

\citet{zhang2025ensemble} & Six-year organic aerosol (OA) source apportionment including two primary OA (POA) factors—hydrocarbon-like OA (HOA) and biomass burning OA (BBOA)—and two secondary OA (SOA) factors—less oxidized oxygenated OA (LO-OOA) and more oxidized oxygenated OA (MO-OOA) in Paris region (2011–2018) & Aerosol chemical speciation monitor (ACSM) organic mass spectral matrix (m/z 13–100), combined with time variables (day of year, day of week, hour of day) from online long-term AMS measurements; meteorological data from ERA5 reanalysis (relative humidity, temperature, pressure) & Six years of ACSM measurements at SIRTA site near Paris, France (November 2011 – March 2018) with PMF-derived OA source factors for model training \\[6pt]

\citet{chen2022machine} & Large-scale cloud microphysical and macrophysical properties including liquid cloud droplet number concentration, effective radius, liquid water path, and cloud fraction over North Atlantic and Arctic regions during the 2014 Holuhraun volcanic eruption period & Satellite observations from MODIS Aqua and Terra (2001–2020) Level 3 cloud optical and microphysical properties; meteorological reanalysis data from ERA5 including 114 variables covering surface to 550~hPa levels; volcanic aerosol perturbation dataset for Holuhraun eruption (September–October 2014) & Combined satellite (MODIS Aqua/Terra) observations and ERA5 reanalysis data from 2001 to 2020 excluding 2014 for training machine learning surrogate model; volcanic eruption period data for analysis and validation \\[6pt]

\citet{li2022projected} & Global near-surface aerosol concentrations (BC, OC, NH$_4^+$, NO$_3^-$, SO$_4^{2-}$, and PM$_{2.5}$) projected for 2015–2100 under multiple SSP emission and climate scenarios & Aerosol species concentrations simulated by GEOS-Chem model (2005–2014) driven by MERRA-2 meteorology; meteorological variables from CMIP6 multi-model monthly simulations including temperature (2~m, 850~hPa, 500~hPa), winds (850~hPa, 500~hPa), precipitation, cloud cover, relative humidity, sea level pressure, and surface shortwave radiation; emissions from CMIP6 SSP scenarios and MEIC for China; land cover, topography, NDVI, and spatiotemporal information & GEOS-Chem simulation data (2005–2014) for training; CMIP6 model meteorological data and SSP emission scenarios for future projections (2015–2100) \\[6pt]

\citet{redemann2024machine} & Aerosol light absorption (ABS) and cloud condensation nuclei (CCN) number concentrations at multiple altitudes and aerosol types sampled across four recent airborne field campaigns over the continental US, Southeast Atlantic Ocean, Philippines, and US Atlantic Coast (2013–2022) & Airborne High Spectral Resolution Lidar-2 (HSRL-2) measurements of aerosol backscatter, depolarization, and extinction at 355, 532, and 1064~nm wavelengths; collocated in situ ABS and CCN measurements; reanalysis data of temperature and relative humidity from ERA5 & Four coordinated airborne field campaigns: DISCOVER-AQ, ACTIVATE, ORACLES, and CAMP2Ex spanning diverse aerosol types including smoke, dust, marine, and pollution aerosols \\[6pt]

\citet{zhao2022simulating} & Monthly sea surface dimethylsulfide (DMS) concentrations over Asian seas in 2017 and associated atmospheric sulfate (SO$_4^{2-}$) aerosol and cloud condensation nuclei (CCN) concentrations, as well as radiative forcing estimates & Twelve environmental predictors including satellite remote sensing data (chlorophyll-a (Chl), photosynthetically available radiation (PAR), particulate inorganic/organic carbon (PIC/POC), diffuse attenuation coefficient at 490~nm (kd490)), oceanographic variables (sea surface temperature, salinity, dissolved oxygen, mixed layer depth), nutrient data (silicate, phosphate, nitrate), and temporal/geographical coordinates; model trained with 4351 seawater DMS measurements from Chinese seas and global datasets using XGBoost machine learning algorithm & GEOS-Chem-TOMAS chemical transport model simulations for 2017 over Asia domain (11°~S to 55°~N, 60°~E to 150°~E); DMS emissions predicted by XGBoost model and validated against cruise survey and long-term site measurements \\[6pt]

\citet{bender2024machine} & Cloud droplet effective radius and cloud albedo over ten geographically distinct regions representing volcanic, anthropogenic, and stratocumulus cloud regimes (2004–2019) & Meteorological parameters from ERA5 reanalysis at 1000, 850, and 700~hPa levels including temperature, relative humidity, geopotential height, zonal and meridional wind, vertical velocity, divergence; air mass origin variables from HYSPLIT backward trajectories (latitude, longitude, altitude); cloud top height and aerosol optical depth (AOD) from MODIS satellite; SO$_2$ column from OMI satellite; surface sulfate (SO$_4$) from MERRA-2 reanalysis & Combined satellite remote sensing (MODIS, OMI), reanalysis datasets (ERA5, MERRA-2), and trajectory data for the ten study regions from October 2004 to December 2019 \\[6pt]

\citet{fuchs2018building} & Low cloud fraction and cloud droplet effective radius in the southeast Atlantic region during biomass burning season (July–September) from 2002 to 2012 & Satellite retrievals of MODIS Aqua Level 3 8-day cloud products (cloud fraction, cloud droplet effective radius), aerosol optical depth (AOD); meteorological reanalysis data from ERA-Interim including lower tropospheric stability, relative humidity at multiple pressure levels, sea surface temperature, temperature at 700~hPa, zonal wind speed at 600~hPa, mean sea level pressure; 5-day backward air-mass trajectories from HYSPLIT model initialized at cloud-top altitude & Combined satellite and ERA-Interim reanalysis data over southeast Atlantic (10–20°~S, 0–10°~E), biomass burning season (July–September), 2002–2012 \\[6pt]

\citet{gettelman2021machine} & Warm rain formation process in global climate simulations including autoconversion and accretion rates of cloud droplets & Detailed quasistochastic bin microphysical model outputs of drop size distributions and process rates; inputs include cloud liquid and rain water mixing ratios, number concentrations, slope, intercept, spectral width parameters, air density, and cloud fraction; neural networks trained to emulate bin model autoconversion and accretion tendencies & CAM6 simulations at 0.9°$\times$1.25° horizontal resolution with 32 vertical levels; training data generated from 2~years of TAU bin microphysical model runs sampled every 25~hours, producing $\sim$1.2 billion samples \\[6pt]

\citet{gong2022understanding} & Ten years of aerosol microphysical properties including particle number size distributions (20~nm–10~\(\mu \rm m\)), light-absorbing carbon (LAC), CCN at various supersaturations, and particle hygroscopicity at Cabo Verde Atmospheric Observatory (CVAO) (2008–2017) & Measurements from TROPOS mobility particle size spectrometer (MPSS), aerodynamic particle sizer (APS), multi-angle absorption photometer (MAAP), cloud condensation nuclei counter (CCNC) at multiple supersaturations; backward air mass trajectories from HYSPLIT model & Long-term in situ measurements at CVAO station, São Vicente Island, Cabo Verde, spanning April 2008 to December 2017 for PNSD and LAC; CCN and hygroscopicity measurements during Oct 2015–Mar 2016 and Sept–Nov 2017 \\[6pt]

\citet{harder2022physics} & Aerosol microphysical tendencies predicted by the M7 aerosol microphysics module in the ECHAM-HAM global climate model & Input variables include aerosol mass and number concentrations for different species and modes, atmospheric state variables such as pressure, temperature, relative humidity; outputs are the predicted tendencies (time derivatives) of aerosol variables over one time step & Training data generated from ECHAM-HAM M7 module simulations at 150~km horizontal resolution, 31 vertical levels, with input–output pairs covering global aerosol microphysical processes; training on January and April days, validation on July, testing on October data \\[6pt]

\citet{jia2024analysis} & Daily marine boundary layer cloud fraction and its sensitivity to cloud droplet number concentration and meteorological factors globally (2011–2019) & Satellite MODIS Terra Level-3 daily cloud products (cloud fraction, cloud droplet effective radius, optical depth, cloud top temperature); ERA5 reanalysis meteorological variables including estimated inversion strength, sea surface temperature, surface sensible and latent heat flux, relative humidity, wind components and vertical velocity at multiple pressure levels; air mass back trajectories & Global dataset from 60°~N to 60°~S, combining MODIS satellite and ERA5 reanalysis data aggregated to 1°$\times$1° grids, with regional XGBoost machine learning models trained on 2011–2016 data and tested on 2017–2019 data \\[6pt]

\citet{marais2020leveraging} & Identification of optically thick aerosols and six cloud types from MODIS and VIIRS multispectral satellite images over ocean (daytime only, 2000s–2010s) & Multispectral reflectance channels (0.555~\(\mu \rm m\), 0.642~\(\mu \rm m\)) and 11~\(\mu \rm m\) brightness temperature from MODIS and VIIRS sensors; convolutional neural networks (CNNs) pretrained on general image datasets and fine-tuned on hand-labeled datasets; spatial texture features extracted from image patches of 25 and 100~pixels & Human-labeled training dataset from adapted NASA Worldview interface; MODIS and VIIRS satellite data patches labeled by expert visual classification, covering various cloud and aerosol regimes over ocean \\[6pt]

\citet{wang2020machine} & Cloud mask classification (clear, liquid water cloud, ice cloud) and cloud thermodynamic phase at 750~m resolution from VIIRS satellite spectral observations (2013–2017) & Spectral reflectance and brightness temperature from VIIRS on Suomi NPP, including 3 infrared bands (8.6, 11, 12~\(\mu \rm m\))) for all-day model and additional 5 VNIR/SWIR bands (0.86, 1.24, 1.38, 1.64, 2.25~\(\mu \rm m\)) for daytime model; ancillary data including surface skin temperature, land surface emissivity, surface types (ocean, forest, cropland, grassland, snow/ice, barren desert, shrubland), solar and satellite viewing geometry; reference labels from collocated CALIOP lidar cloud layer products & Training and validation on collocated VIIRS and CALIOP observations from 2013 to 2017 globally, covering seven surface types and wide viewing zenith angle range \\[6pt]

\citet{yorks2021aerosol} & Vertical profiles of aerosol and cloud layers, including layer detection and cloud-aerosol discrimination at 5~km horizontal resolution from Cloud-Aerosol Transport System (CATS) lidar onboard the International Space Station (ISS) (2015–2017) & Attenuated total backscatter and volume depolarization ratio at 1064 nm wavelength from CATS L1B data; ancillary meteorological reanalysis data from NASA MERRA-2; machine learning techniques including wavelet denoising and convolutional neural networks (CNN) trained on CATS Level 0 and Level 2 products for enhanced daytime aerosol-cloud detection and classification & CATS Level 0 and Level 2 data products collected from March 2015 to October 2017 onboard ISS, spanning multiple geographic regions and diurnal cycles \\[6pt]

\citet{zhao2024studying} & Aerosol indirect effects on deep convective cloud microphysical and macrophysical properties including cloud particle effective radius, cloud optical depth, ice water path, cloud cover fraction, cloud top height, and cloud top temperature over global oceans (1982–2019) & Satellite climate data records: NOAA AVHRR Aerosol Optical Thickness (AOT) CDR v4.0 (1982–2019) providing aerosol index (AIX); NOAA AVHRR + HIRS PATMOS-x cloud CDR v6.0 (1982–2019) for cloud properties; meteorological fields (19 variables) from NCEP Climate Forecast System Reanalysis (CFSR) monthly means & Global long-term satellite observations (1982–2019) combined with NCEP CFSR reanalysis data; machine learning models including XGBoost nonlinear regression, SHapley Additive exPlanations (SHAP) for interpretability, back-propagation neural networks for nonlinear fitting, and singular value decomposition for coupled variability analysis \\[6pt]

\citet{hughes2018machine} & Global spatial distribution of aerosol mixing state metric \(\chi\) resolved by particle size and season (2010) & Input features include 34 variables from GEOS-Chem-TOMAS global chemical transport model output: gas-phase species concentrations (SO$_2$, NO$_x$, NH$_3$, VOCs), aerosol species mass and number concentrations (sulfate, organic aerosol, black carbon, sea salt, dust), solar zenith angle, latitude, meteorological variables & Training dataset generated from 1000 particle-resolved PartMC-MOSAIC box model simulations with diverse atmospheric scenarios; testing on 240 independent scenarios; applied globally on GEOS-Chem-TOMAS 2010 simulation fields with 2°$\times$2.5° horizontal resolution and 47 vertical layers \\[6pt]

\citet{jiang2025integrating} & Aerosol mixing state index \(\chi\) for 100–700~nm aerosol particles & Particle-resolved aerosol model PartMC-MOSAIC simulations generating aerosol species mass concentrations and environmental variables (temperature, relative humidity, gas species, VOCs, aerosol species including OA, BC, NH$_4$, SO$_4$, NO$_3$); observational fine-tuning data from MEGAPOLI winter campaign with aerosol time-of-flight mass spectrometer (ATOFMS) & Pretraining on 1000 PartMC-MOSAIC simulated scenarios (each with 24~hourly snapshots) with Latin Hypercube Sampling of emission and meteorological parameters; fine-tuning using MEGAPOLI campaign observational hourly data (Paris, January–February 2010) \\[6pt]

\citet{shen2024improving} & Black carbon (BC) mixing state index \(\chi\), mass ratio of coating to BC ($R_{\rm BC}$), BC hygroscopicity, BC activation fraction, and BC concentration in accumulation mode aerosol & Particle-resolved simulations from PartMC-MOSAIC used to train an XGBoost machine learning model predicting BC mixing state index; input parameters include aerosol bulk concentrations (BC, dust, sea salt, primary organic matter, secondary organic aerosol, sulfate), gas mixing ratios (DMS, H$_2$O$_2$, H$_2$SO$_4$, O$_3$, SOAG, SO$_2$), and environmental variables (air temperature, relative humidity, solar zenith angle); coupled online with CAM6-MAM4 modal aerosol model at 0.9°$\times$1.25° resolution & CAM6-Chem model simulations for year 2011 using MEIC emissions over China region and historical CMIP6 emissions elsewhere; evaluation against surface observations of $R_{\rm BC}$ and BC concentrations from multiple sites globally \\[6pt]

\citet{zheng2021estimating} & Three aerosol mixing state indices for submicron aerosol globally: chemical species abundance mixing state ($\chi_a$), mixing of optically absorbing and nonabsorbing species ($\chi_o$), and mixing of hygroscopic and nonhygroscopic species ($\chi_h$) & Bulk aerosol species concentrations, gas phase species, and environmental variables from Community Earth System Model (CESM) with Modal Aerosol Module (MAM4); features include black carbon, dust, sea salt, primary organic matter, secondary organic aerosol, sulfate, dimethyl sulfide, hydrogen peroxide, sulfuric acid, ozone, semi-volatile organic gas, sulfur dioxide, temperature, relative humidity, solar zenith angle & Training data from 2000 PartMC-MOSAIC particle-resolved aerosol model scenarios sampled using Latin Hypercube over diverse emission and meteorological conditions; applied to CESM2 historical simulation outputs at 0.9°$\times$1.25° resolution (1970–2014) \\[6pt]

\citet{zheng2021quantifying} & Global distribution and structural uncertainty of aerosol mixing state indices $\chi_o$, $\chi_c$, and $\chi_h$ based on optically absorbing/non-absorbing, primary carbonaceous/non-primary carbonaceous, and hygroscopic/non-hygroscopic aerosol species mixing & Aerosol species mass concentrations from CESM2-CAM6 with MAM4 modal aerosol module (four lognormal modes: Aitken, accumulation, coarse, primary carbon); surrogate species grouping for calculating mixing state indices; benchmark machine-learned model (XGBoost) trained on 45,000 PartMC-MOSAIC particle-resolved simulations & Global CESM2 historical simulation outputs for 2011 at $\sim$1° horizontal resolution; ML model trained on particle-resolved model data representing diverse emission and meteorological scenarios \\[6pt]

\citet{bao2023retrieval} & Aerosol optical depth (AOD) at multiple spectral bands (0.46, 0.64, 2.30~\(\mu \rm m\)) and Angstrom exponent (AE) over East Asia and adjacent oceanic regions using Himawari-8 satellite (2018) & Top-of-atmosphere (TOA) reflectance at 0.46, 0.64, and 2.30~\(\mu \rm m\) from Himawari-8 Advanced Himawari Imager (AHI); differential operator based on radiative transfer theory to quantify linear relationship between AOD and TOA reflectance enhancement; random forest model trained with spectral reflectance, solar and sensor geometry, aerosol optical properties (refractive index, single scattering albedo), and aerosol microphysical parameters (median radius and geometric standard deviation for fine and coarse modes) & Training and validation data include Himawari-8 L1B images and AERONET version 3 level 2 AOD and AE measurements from 2008 to 2018; sample-based and site-based tenfold cross-validation used for model assessment; aerosol types clustered by k-means for regional aerosol characterization \\[6pt]

\citet{kumar2022correcting} & Particle-phase aerosol light absorption coefficient (B$_\mathrm{abs}$) at 467, 530, and 660~nm wavelengths at the Atmospheric Radiation Measurement Southern Great Plains (SGP) site (June–September 2015) & Filter-based absorption coefficients and transmission from Particle Soot Absorption Photometer, aerosol scattering coefficients from nephelometer, non-refractory aerosol mass concentrations from Aerosol Chemical Speciation Monitor; random forest regression machine learning model trained on these measurements to predict corrected B$_\mathrm{abs}$ & High-time-resolution ambient aerosol data from ARM SGP site over 3 months; reference particle-phase B$_\mathrm{abs}$ measured by co-located three-wavelength Photoacoustic Absorption Spectrometer; additional validation on laboratory combustion aerosol datasets \\[6pt]

\citet{luo2018applying} & Integral optical properties of black carbon (BC) fractal aggregates including extinction efficiency (Q$_\mathrm{ext}$), absorption efficiency (Q$_\mathrm{abs}$), scattering efficiency (Q$_\mathrm{sca}$), and asymmetry factor over a wide range of wavelengths (300–3100 nm) and fractal morphology parameters & Fractal parameters of BC aggregates (fractal dimension $D_f$, prefactor $k_0$, number of monomers $n_s$, monomer radius $a$), refractive index fixed at 1.95 + 0.79i; training data generated from numerically exact multiple-sphere T-matrix method (MSTM) simulations; support vector machine regression with radial basis function kernel to model nonlinear relationships between morphology and optical properties & Training dataset from MSTM calculations covering ranges: $D_f$=1.8–2.2, $k_0$=1.2–1.6, $n_s$=10–600, wavelength 300–3100 nm; validation against MSTM outputs and extension to different monomer radii and refractive indices \\[6pt]

\citet{berhane2024comprehensive} & Monthly Aerosol Optical Depth (AOD) and aerosol species over the Middle East and North Africa region (2003–2020) & MODIS combined Dark Target and Deep Blue monthly AOD at 1° resolution, CAMSRA and MERRA-2 reanalysis aerosol species AOD (dust, black carbon, organic carbon, sulfate, sea salt) at 0.75° and 0.5° resolutions respectively, meteorological drivers & Monthly AOD and aerosol species data from MODIS, CAMSRA, and MERRA-2 reanalysis datasets spanning 2003–2020 \\[6pt]

\citet{huttunen2016retrieval} & Aerosol Optical Depth (AOD) retrieval from surface solar radiation (SSR) measurements at Thessaloniki, Greece (2005–2008) & Surface solar radiation from pyranometer, water vapor content (WVC), solar zenith angle (SZA) measurements, compared with AERONET sun photometer AOD data & AERONET Level 2.0 AOD measurements for validation; Training data SSR, WVC, SZA from 2009–2014; Validation data from 2005–2008 \\[6pt]

\citet{just2018correcting} & Measurement error in MAIAC satellite Aerosol Optical Depth (AOD) over Northeastern USA (2000–2016) & 
MAIAC AOD from MODIS Aqua and Terra satellites at 1~km resolution (2000–2016), 52 predictor variables including MAIAC quality control flags (e.g., AOT uncertainty, relative azimuth, cloud and adjacency masks), meteorological variables from NCEP reanalysis (air temperature, evaporation, planetary boundary layer height, surface pressure, precipitable water, specific humidity, wind components, visibility), land use variables (elevation from SRTM, forest and water cover from NLCD), spatial features (distance to missing data edges, moving window statistics over multiple spatial scales), regional variables (ecoregion and political region AOD statistics), cluster variables (contiguous non-missing AOD clusters), temporal variables (date integer and bimonthly indicator) & 
Collocated ground-based AERONET AOT measurements from 79 stations (2000–2015) for model training and validation \\[6pt]

\citet{lary2009machine} & Bias correction of MODIS Aerosol Optical Depth (AOD) retrievals globally (2002–2008) & MODIS Terra and Aqua Collection 5 AOD at 550~nm, surface type classification (GLC2000), solar and sensor zenith and azimuth angles, scattering angle, reflectance at 550~nm & Collocated ground-based AERONET AOD measurements within 30~minutes of satellite overpass globally (2002–2008) \\[6pt]

\citet{nabavi2018prediction} & Monthly mean Aerosol Optical Depth (AOD) over West Asia region (April–September, 2003–2013) & MODIS Aqua Deep Blue (DB) AOD at 10~km resolution, ECMWF ERA-Interim reanalysis meteorological variables (10~m wind speed, vertical velocity at 850 hPa, soil temperature, surface albedo), ESA-CCI soil moisture, GPCC precipitation, GIMMS NDVI, SPEI drought index, West Asia dust source function, and area-averaged AOD over main dust sources & MODIS DB AOD monthly means from Aqua satellite for 2003–2013, with training on 2003–2010 and testing on 2011–2013 periods \\[6pt]

\citet{yeom2021estimation} & Hourly Aerosol Optical Depth (AOD) over Northeast Asia (2016–2018) & GOCI geostationary satellite top-of-atmosphere (TOA) multispectral reflectances at 412, 443, 490, 555, 660, and 680~nm bands aggregated to 5$\times$5 km resolution, solar zenith and azimuth angles & Ground-based AERONET Level 2.0 AOD measurements from 33 stations across Northeast Asia (2016–2018) \\[6pt]

\citet{logothetis2023aerosol} & Aerosol optical depth (AOD) at 440, 500, and 675~nm, Ångström Exponent (AE) between 440–675~nm, Fine Mode Fraction at 500~nm; aerosol type classification into six classes (biomass burning, continental, dust, marine, mixed, polluted) & Red-Green-Blue channel intensities and sun-saturated area from a Mobotix Q24M all-sky imager (ASI); solar zenith angle; total column water vapor (TCWV) from CAMS reanalysis; supervised machine learning with Light Gradient Boosting Machine & Dataset of 3212 synchronized ASI images and AERONET Level 2.0 Version 3 retrievals from National Observatory of Athens, Greece (Jan–Nov 2021) \\[6pt]

\citet{barahona2025deep} & Aerosol size distribution and mixing state including modal number concentrations and mass of 31 aerosol tracers (e.g., sulfate, sea salt, dust, organics, black carbon) & Bulk mass mixing ratios of dust, sulfates, organics, black carbon, sea salt; Atmospheric state variables: temperature (T) and air density ($\rho_{air}$) & Synthetic dataset generated from 5-year GEOS+MAM simulation at 1° resolution, 72 vertical levels, instantaneous outputs at 9:00 and 21:00~UTC; 25 files for training, 10 files for testing; plus observational evaluation data from 24 European aerosol monitoring sites (2008–2009) \\[6pt]

\citet{ren2020prediction} & Aerosol particle size distribution over the whole atmosphere column (52 size channels from 0.5 to 20~\(\mu rm m\)) & Aerosol optical depth (AOD) at eight wavelengths (340, 380, 440, 500, 670, 870, 1020, 1640~nm) measured by CE-318 sun photometer; extinction efficiency factors from Mie scattering theory at corresponding wavelengths & 500 sets of measured data from CE-318 sun photometer and APS 3321 aerodynamic particle sizer, collected from June to July 2019 at North Minzu University, China (including sunny, winter sunny, and dusty days)\\[6pt]

\citet{wu2025estimating} & Atmospheric aerosol number size distribution & Meteorological variables (wind speed, wind direction, temperature, relative humidity, pressure, radiation), trace gases (NO$_x$, SO$_2$, CO, O$_3$), total particle number concentration (N$_{tot}$) & Measurement data from SMEAR I, II, and III stations in Finland (2005–2019) and Qvidja station (2019 test data), obtained using DMPS and APS instruments \\[6pt]

\citet{govindasamy2021machine} & Global solar radiation prediction influenced by PM$_{10}$ aerosol concentration & Meteorological variables including temperature, relative humidity, wind speed and direction, atmospheric pressure, rainfall; PM$_{10}$ aerosol concentration measurements & Ground-based meteorological and PM$_{10}$ aerosol data collected from multiple monitoring stations across South Africa \\[6pt]

\citet{jin2019machine} & Non-dust PM$_{10}$ concentration (bias correction) & Air quality indices: PM$_{2.5}$, SO$_2$, NO$_2$, CO, O$_3$; meteorological data; PM$_{2.5}$ measurements at nearby sites (up to 3 within 80~km radius) & Two years of historical air quality and meteorological data from China MEP monitoring network (1351 stations) \\

\citet{johnson2018using} & PM$_{2.5}$ concentration measurements corrected for low-cost sensor biases & Raw sensor data including particle counts from low-cost aerosol monitors, meteorological variables (temperature, relative humidity), and co-located reference instrument measurements & Field data collected from low-cost aerosol monitors deployed in a dense urban environment with collocated reference instruments for training and validation \\[6pt]

\citet{shi2022aerosol} & Aerosol iron solubility (fractional solubility of iron in aerosols) globally & Aerosol chemical composition (total Fe, sulfate, organic carbon, nitrate, dust, sea salt, ammonium), meteorological variables (temperature, relative humidity, wind speed), aerosol physical properties (mass concentration, particle size), geographical and temporal variables (latitude, longitude, month), aerosol source region classification & Global observational aerosol iron solubility datasets from multiple field campaigns and measurement networks, combined with satellite and reanalysis meteorological and aerosol data covering various marine atmospheric regions globally over multiple years \\[6pt]

\citet{song2023lightning} & Lightning occurrence and intensity nowcasting & Aerosol optical depth (AOD) from satellite, meteorological variables including temperature, humidity, wind speed and direction, atmospheric instability indices, cloud properties from satellite observations & Multi-source dataset combining GOES-16 satellite data, ground-based lightning detection networks, and aerosol retrievals over the contiguous United States for multiple convective seasons (2018–2021) \\[6pt]

\hline
\end{longtable}
\end{landscape}

\section{Air quality rlated works}
\citet{reid2015spatiotemporal} applied generalized boosting models (GBM) to predict spatiotemporal variations of PM$_{2.5}$ concentrations during the 2008 Northern California wildfires. By integrating diverse data sources including satellite aerosol optical depth, chemical transport model outputs, meteorological variables, and land use characteristics, the study achieved a high cross-validated $R^{2}$ of 0.80, demonstrating strong predictive capability. Their approach effectively captured the complex spatial and temporal dynamics of wildfire smoke plumes, addressing challenges posed by sparse monitoring data. This work highlights the utility of machine learning algorithms to improve air pollution exposure estimates in wildfire-affected regions, providing valuable tools for epidemiological research and public health interventions.

\citet{chen2018machine} demonstrated the random forests model can significantly improve the estimation of ground-level PM$_{2.5}$ concentrations by integrating satellite-derived aerosol optical depth, meteorological data, and land use information. Their study showed that random forests models outperformed traditional regression methods in capturing the spatiotemporal variability of PM$_{2.5}$ across China at daily, monthly, and seasonal scales. The model’s high predictive accuracy and ability to incorporate diverse predictors make it a robust tool for long-term air quality assessment and health impact studies. This work highlights the growing potential of machine learning techniques in environmental exposure modeling, providing valuable insights for air pollution management and policy development.

\citet{xiao2018ensemble} developed a regional ensemble machine learning framework to accurately estimate historical daily PM$_{2.5}$ concentrations across China from 2013 to 2017 using satellite aerosol optical depth (AOD), meteorological, and land-use data. By partitioning China into seven temporally stable clusters based on geographically weighted regression and training separate models—including random forest, extreme gradient boosting, and generalized additive models—in each cluster, their approach effectively controlled spatial heterogeneity and improved prediction performance. The ensemble model achieved a cross-validation \(R^2\) of 0.79 with a root mean square error (RMSE) of 21 \(\mu g/m^3\), outperforming national and single-model approaches. Importantly, the model demonstrated robust hindcast capabilities, accurately predicting PM$_{2.5}$ levels in Beijing during 2008, well before the training period. This study highlights the value of combining regional clustering with ensemble machine learning to reconstruct long-term, high-resolution air pollution records in regions lacking extensive ground monitoring, thereby supporting epidemiological studies and air quality management.

\citet{xu2018evaluation} evaluated the performance of eight machine learning algorithms for estimating monthly ground-level PM$_{2.5}$ concentrations across British Columbia using multiple remote sensing datasets, including satellite-derived aerosol optical depth (AOD) and meteorological reanalysis data. Their study showed that ensemble methods such as Cubist, random forest, and XGBoost outperformed traditional linear models, with Cubist achieving the best accuracy characterized by a cross-validated root mean square error (CV-RMSE) of 2.64 $\mu$g/m$^3$ and a coefficient of determination (CV-$R^{2}$) of 0.48. Variable importance analysis identified monthly AOD and elevation as key predictors, highlighting the influence of complex terrain on PM$_{2.5}$ distribution. The study also pointed out challenges such as data gaps due to snow cover affecting AOD retrieval and limitations in spatial resolution. Overall, this work demonstrates that appropriate machine learning algorithms combined with multi-source remote sensing data can effectively improve PM$_{2.5}$ estimation in regions with complex topography and sparse monitoring networks.

\citet{chen2019stacking} developed a stacking machine learning model that combined AdaBoost, XGBoost, and random forest algorithms through multiple linear regression to estimate hourly PM$_{2.5}$ concentrations in Central and Eastern China based on high-temporal-resolution Himawari 8 aerosol optical depth data. The stacking model outperformed individual submodels, achieving an overall coefficient of determination (R$^2$) of 0.85 and a root mean square error (RMSE) of 17.3 \(\mu g/m^3\), with peak performance around 14:00 local time (R$^2$ = 0.92). Seasonal and spatial analyses showed better model stability during polluted seasons (autumn and winter) and higher accuracy in densely monitored central and eastern regions. Their results highlight the effectiveness of stacking ensemble methods combined with geostationary satellite data in providing detailed hourly air quality estimates, which is critical for pollution control and health risk assessment.

\citet{karimian2019evaluation} evaluated the performance of three machine learning models, multiple additive regression trees, deep feedforward neural networks, and a hybrid long short-term memory network, for forecasting PM$_{2.5}$ concentrations in Tehran, Iran. Their results demonstrated that the LSTM model, which captures temporal dependencies in sequential data, outperformed the other methods by achieving the lowest root mean square error (RMSE) of 8.91 \(\mu g/m^3\), mean absolute error (MAE) of 6.21 \(\mu g/m^3\), and the highest coefficient of determination (R$^2$) of 0.80 for 48-hour forecasts. The study highlights the importance of modeling temporal dynamics for accurate air quality predictions in highly polluted urban environments and suggests that deep learning approaches like LSTM can effectively improve short-term PM$_{2.5}$ forecasting.

\citet{li2019predicting} proposed a hybrid remote sensing and machine learning approach named RSRF, which integrates high-quality MODIS Aqua Deep Blue aerosol optical depth (AOD), meteorological variables, and air pollutant concentrations to estimate daily ground-level PM$_{2.5}$ concentrations across the Beijing-Tianjin-Hebei (BTH) region. The RSRF model, based on random forest regression, outperformed traditional methods such as multiple linear regression, multivariate adaptive regression splines, and support vector regression, achieving a cross-validated \( R^2 \) of 0.843 and a root mean square error (RMSE) of 25.32 \(\mu g/m^3\). Seasonal analyses revealed that the model prediction accuracy varied with season and was generally lower in summer due to fewer samples and complex atmospheric conditions. Variable importance and sensitivity analyses identified AOD, NO$_2$, and CO as key predictors, while meteorological variables played a secondary but essential role. Application of the RSRF model to a severe haze pollution episode in winter 2014 demonstrated its capability to capture spatiotemporal variations and provided insights into pollution sources and transport. This study highlights the effectiveness of combining remote sensing and machine learning techniques for regional air quality assessment and supports targeted emission control strategies in complex environments.

\citet{liu2019satellite} developed a random forest-based machine learning model that directly uses top-of-atmosphere (TOA) reflectances from the Himawari-8 geostationary satellite, combined with meteorological parameters, to estimate hourly ground-level PM$_{2.5}$ concentrations across China in 2016. Their Ref-PM$_{2.5}$ model achieved a cross-validated coefficient of determination (CV-$R^2$) of 0.86 and a root mean square error (RMSE) of 17.3 \(\mu g/m^{3}\) at the hourly scale, comparable to a model based on satellite-retrieved aerosol optical depth (AOD). The study demonstrated that using TOA reflectance directly circumvents some uncertainties inherent in AOD retrievals and increases sample size, enhancing spatial and temporal coverage. Seasonal and regional analyses revealed that the model captured spatial patterns and diurnal cycles well, especially in highly polluted and densely monitored regions such as Beijing-Tianjin-Hebei, Yangtze River Delta, and Pearl River Delta. This work highlights the effectiveness of ensemble machine learning methods applied to satellite reflectance data for improving air quality monitoring and pollution episode tracking.

\citet{chen2020estimating} developed an improved random forest (IRF) model incorporating the newly released 1-km resolution MODIS aerosol optical depth (AOD) data alongside meteorological variables to estimate daily PM$_{2.5}$ concentrations over Shenzhen, China. The IRF model effectively captured the complex nonlinear relationships and spatiotemporal heterogeneities inherent in coastal urban environments, outperforming traditional geographically and temporally weighted regression (GTWR) and standard random forest (RF) models. It achieved a high overall coefficient of determination (R$^{2}$) of 0.915 and a root mean square error (RMSE) of 3.66 \(\mu g/m^{3}\), demonstrating superior accuracy at daily, seasonal, and annual scales. This study underscores the value of integrating high-resolution remote sensing data with advanced machine learning techniques for urban air quality monitoring and offers a promising tool for detailed PM$_{2.5}$ spatial distribution mapping in rapidly developing coastal regions.

\citet{just2020advancing} presented a rigorous machine learning framework employing extreme gradient boosting (XGBoost) with spatially explicit cross-validation to predict daily PM$_{2.5}$ concentrations at 1~km resolution across the Northeastern United States from 2000 to 2015. Their approach addresses critical challenges of overfitting and data leakage inherent in spatiotemporal environmental data by implementing a spatial clustering strategy to ensure robust model evaluation away from dense monitoring networks. Through recursive feature selection guided by Shapley Additive Explanation (SHAP) values, they developed a parsimonious model with only eight predictors, balancing interpretability and predictive power. The model achieved a spatial cross-validation \(R^{2}\) of 0.76 and root mean squared error around 3.5 \(\mu g/m^{3}\), demonstrating strong performance in diverse urban, suburban, and rural settings. This study underscores the importance of spatially aware validation and model interpretability in air pollution exposure assessment, providing a scalable and reliable tool for epidemiological applications and environmental policy.

\citet{stirnberg2020mapping} developed a parsimonious, high-resolution (1~km $\times$ 1~km) daily PM$_{2.5}$ prediction model for the Northeastern USA from 2000 to 2015, employing an extreme gradient boosting (XGBoost) machine learning algorithm with rigorous spatial cross-validation to avoid overfitting due to spatial autocorrelation and data leakage. Their model incorporated a carefully selected set of predictors, including corrected MAIAC aerosol optical depth (AOD) from Aqua and Terra satellites, planetary boundary layer height, geographic coordinates, date, inverse distance weighted (IDW) PM$_{2.5}$ surfaces, and land development metrics. The study highlighted the critical importance of spatially explicit model evaluation methods, demonstrating that traditional random cross-validation tends to underestimate prediction errors. The final model achieved robust performance with spatial cross-validated \( R^{2} \) around 0.70 and root mean square error (RMSE) near 3.2 \(\mu g/m^{3}\), representing a significant advancement in exposure assessment for epidemiological applications. Additionally, the authors utilized Shapley Additive Explanations (SHAP) to interpret feature contributions, enhancing model transparency and supporting future efforts to balance complexity, interpretability, and accuracy in air pollution modeling.

\citet{minh2021pm2} developed a machine learning-based PM$_{2.5}$ forecasting system for Ho Chi Minh City, Vietnam, integrating the Weather Research and Forecasting (WRF) model meteorological simulations with six regression algorithms. Among these, the Extra Trees Regression model achieved the best performance, with an \( R^{2} \) of 0.68, RMSE of 7.68 \(\mu g/m^{3}\), MAE of 5.38 \(\mu g/m^{3}\), and classification accuracy of 74\% based on the U.S. EPA PM$_{2.5}$ breakpoints. The study demonstrated that PM$_{2.5}$ predictions using WRF-simulated meteorological data closely matched those using observed data, supporting the viability of this approach for short-term (48 hours) and medium-term (7 days) air quality forecasting in urban Southeast Asian settings. The authors highlighted the model’s practical potential for pollution warning systems and recommended future enhancements through higher-frequency data and ensemble methods to improve forecast accuracy.

\citet{gupta2021machine} developed a random forest machine learning algorithm integrating NASA's MERRA2 reanalysis aerosol and meteorological data with hourly ground-level PM$_{2.5}$ observations from 51 stations across Thailand for 2018. The model demonstrated excellent performance, achieving a correlation coefficient of 0.95 and a near-zero mean bias between estimated and observed PM$_{2.5}$ concentrations at hourly and daily scales. Despite minor underestimation during very clean ($<$ 10 \(\mu g/m^{3}\)) and highly polluted ($>$ 80 \(\mu g/m^{3}\)) conditions, the model effectively captured seasonal and diurnal PM$_{2.5}$ variability. The study highlights the capacity of machine learning approaches to correct biases in physical model outputs, providing high-resolution, bias-corrected PM$_{2.5}$ datasets valuable for air quality assessment, trend analysis, and health impact studies in Southeast Asia.

\citet{chen2022obtaining} developed an interpretable machine learning model based on Extra Trees (ET) to estimate the vertical distribution of PM$_{2.5}$ concentrations across five regions in China using CALIOP aerosol optical depth (AOD) data at different altitudes combined with meteorological variables and ground station observations from 2015 to 2019. Their model achieved a high overall coefficient of determination (\( R^{2} = 0.85 \)) and root mean square error (RMSE) of 17.77 \(\mu g/m^{3}\), with 73\% of monitoring sites exhibiting \( R^{2} > 0.7 \). The study revealed that the optimal altitude layer AOD contributed more significantly to PM$_{2.5}$ estimation than total-column AOD, and that seasonal and regional variations influenced model performance. Vertical profiles showed that PM$_{2.5}$ concentrations were highest near the ground and decreased rapidly with altitude, with notable seasonal differences in vertical dispersion. A declining trend in PM$_{2.5}$ vertical concentrations from 2015 to 2019 was observed in major urban regions, attributed to both meteorological factors and emission control measures. This work demonstrates the value of integrating satellite lidar data and machine learning to characterize the three-dimensional structure of air pollution, providing critical insights for air quality management and pollution mitigation strategies.

\citet{kulkarni2022model} conducted a comprehensive comparison of ten statistical and machine learning models to predict daily mean PM$_{2.5}$ concentrations from high-resolution (1~km) MODIS-MAIAC aerosol optical depth (AOD) data combined with meteorological, land-use, and ancillary variables over the National Capital Territory (NCT) of Delhi in 2019. The study found that machine learning models, particularly XGBoost, outperformed statistical models with the highest cross-validated coefficient of determination (R$^2$) of 0.93 and normalized root mean square error (NRMSE) of 0.18. Among statistical models, linear mixed-effects models incorporating day- and site-specific random effects showed strong performance (R$^2$ = 0.91). The inclusion of meteorological parameters and land-use proxies significantly improved model accuracy over univariate approaches relying on AOD alone. The results emphasize the importance of accounting for spatiotemporal variability and complex nonlinear relationships in the AOD-PM$_{2.5}$ association, especially in regions with heterogeneous pollution sources and meteorological conditions like Delhi. This work provides critical guidance on model selection for PM$_{2.5}$ estimation using satellite data, highlighting the advantages of advanced machine learning techniques in capturing fine-scale air pollution patterns.

\citet{lee2021air} developed a novel approach that integrates machine learning (random forest) and three-dimensional variational data assimilation (3D-VAR) to improve air quality forecasts over East Asia by estimating ground-level PM$_{10}$ and PM$_{2.5}$ concentrations from satellite aerosol optical depth (AOD) data. Their method replaces the conventional AOD observation operator with a machine learning-derived PM operator, significantly reducing observation errors and providing more accurate initial conditions for numerical air quality models. The combined ML-DA system (MLDA) showed marked improvements in PM forecast accuracy, with correlation coefficients reaching 0.9 and forecast skill enhancements lasting over 24 hours for PM$_{10}$ and up to 6 hours for PM$_{2.5}$. The study also demonstrated that MLDA effectively mitigates spatial coverage limitations of ground-based PM monitoring by leveraging satellite data, improving predictions in unmonitored regions. These results highlight the potential of synergistically combining machine learning with data assimilation to enhance satellite-based air quality forecasting, particularly in complex regional environments.

\citet{karimian2023evaluation} investigated the impact of geostationary Himawari-8 aerosol optical depth (AOD) data on PM$_{2.5}$ prediction accuracy over the Yangtze River Delta region and proposed a novel hybrid forecasting framework combining wavelet transform, machine learning techniques (random forest, GBRT, XGBoost), and minimum redundancy maximum relevance (mRMR) feature selection. They found that high AOD missing rates, particularly in western inland areas and during spring, limit the usefulness of AOD as an auxiliary predictor, with inclusion of AOD sometimes slightly reducing model accuracy. The hybrid model incorporating wavelet decomposition and mRMR significantly outperformed baseline models, achieving an average \( R^{2} \) improvement of 11.64\% and substantial reductions in RMSE, MAE, and MAPE. Spatial variability in model performance was observed across study sites, with the highest accuracy in coastal regions. This work highlights the benefits of integrating advanced data processing and machine learning approaches to improve PM$_{2.5}$ forecasting in regions with complex atmospheric conditions and data limitations, and provides a framework applicable to other air pollution prediction contexts.

\section{Cloud condensation nuclei related works}
\citet{nair2020using} developed a random forest regression model (RFRM) to estimate cloud condensation nuclei number concentrations at 0.4\% supersaturation ([CCN$_{0.4}$]) from widely available atmospheric state and composition variables, including PM$_{2.5}$ speciation, gaseous species, and meteorological parameters. Trained on 30 years of chemical transport model simulations with detailed microphysics (GEOS-Chem-APM), the RFRM demonstrated strong predictive capability with a median mean fractional bias of 4.4\%, Kendall’s rank correlation coefficient (\(\tau\)) of approximately 0.88, and 96.3\% of predictions within a good agreement range (|MFB| $<$ 0.6) across diverse global locations and altitudes. The model identified PM$_{2.5}$ inorganic fractions and key trace gases such as SO$_2$ and NO$_x$ as the most important predictors. Application of the RFRM to real measurements at the ARM Southern Great Plains site, using a reduced set of available predictors, yielded robust performance despite data limitations. This study illustrates the potential of machine learning techniques to provide accurate CCN estimates where direct measurements are sparse, supporting improved aerosol–cloud interaction representation in climate and Earth system models.

\citet{nair2021machine} demonstrated that machine learning models, specifically random forest regression, can accurately derive cloud condensation nuclei (CCN) number concentrations from aerosol chemistry and meteorology data without explicit size distribution inputs. Using extensive multi-campaign airborne measurements spanning varied physicochemical regimes across the troposphere, the study revealed that aerosol mass speciation and atmospheric variables implicitly contain aerosol size distribution information critical for CCN quantification. Their interpretable and robust machine learning framework, grounded in physicochemical principles, offers a computationally efficient pathway to improve CCN representation in global climate models. This approach has significant potential to reduce uncertainties in aerosol-cloud interactions, thereby enhancing confidence in anthropogenic forcing assessments and climate change projections.

\citet{liang2022prediction} applied a random forest machine learning model to estimate cloud condensation nuclei (CCN) spectral parameters based on extensive aerosol observational data collected during four field campaigns in the North China Plain. Their results showed that the model trained on data from one campaign could successfully predict CCN parameters in other campaigns, with coefficients of determination around 0.5, despite variations in aerosol properties and measurement uncertainties across campaigns. The study identified mass concentration of black carbon and hemispheric backscattering fraction as the most important input variables for predicting CCN spectral parameters, highlighting the significant role of aerosol optical properties over chemical composition. The authors also demonstrated that models trained solely on aerosol optical or chemical data performed worse than those using combined datasets, emphasizing the value of optical measurements in estimating CCN activity. This work underscores the potential of machine learning approaches to improve CCN predictions and supports their application in cloud microphysical parameterizations and climate models.

\citet{park2023new} developed a novel cloud condensation nuclei number concentration (N$_{\rm CCN}$) prediction method, termed MLRNMF, which combines multiple linear regression (MLR) with non-negative matrix factorization (NMF) applied to aerosol number size distribution data measured over the Korean Peninsula. The method demonstrated strong predictive performance with coefficients of determination (\( R^2 \)) of 0.81 and 0.71 for data sets from the Yellow Sea and Seoul, respectively, and met established performance goals for mean fractional bias (MFB) and mean fractional error (MFE). Compared to the traditional backward integration method, MLRNMF showed improved robustness, particularly in handling measurement uncertainties and accounting for external aerosol mixing states, which are common in complex urban and marine environments. The study also confirmed the capability of MLRNMF to generalize across seasonal variations and different sampling periods separated by several years, underscoring its potential to generate abundant, reliable N$_{\rm CCN}$ data critical for reducing uncertainties in climate change predictions.

\citet{rejano2024ccn} conducted an extensive study on cloud condensation nuclei (CCN) estimations at a high-altitude remote site in the Sierra Nevada, focusing on the role of organic aerosol (OA) variability and hygroscopicity. Using positive matrix factorization (PMF), they identified three OA factors—hydrocarbon-like OA (HOA), less-oxidized oxygenated OA (LO-OOA), and more-oxidized oxygenated OA (MO-OOA)—with secondary organic aerosol dominating the aerosol population. The study evaluated several OA hygroscopicity (\(\kappa_{OA}\)) schemes to predict CCN concentrations, finding that while all approaches showed reasonable closure with observations, none fully captured the diurnal variability, particularly during new particle formation and vertical transport events. To improve predictions, they developed a neural network model incorporating particle number concentration, OA fraction, oxidation degree (f44), and solar irradiance, which reduced bias and better reproduced observed CCN variability throughout the day. This work highlights the complex influence of OA composition and oxidation on CCN activity at remote mountain sites and demonstrates the potential of machine learning to enhance CCN predictions, thereby reducing uncertainties in aerosol-cloud interactions and climate modeling.

\citet{yu2022use} developed a Random Forest Regression Model (RFRM) trained on long-term global simulations from a size-resolved aerosol microphysics model to improve particle number concentration (PNC) predictions in the GISS-ModelE2.1 climate model while maintaining computational efficiency. The RFRM significantly enhanced agreement with observations worldwide, capturing complex dependencies of PNC on aerosol mass, meteorology, and chemical variables beyond conventional bulk aerosol schemes. Implementation of the RFRM reduced the aerosol indirect radiative forcing (RF$_{\rm aci}$) magnitude from -1.46 to -1.11 W·m$^{-2}$, bringing model estimates closer to IPCC median values and reducing uncertainties related to cloud droplet number concentration changes induced by anthropogenic emissions. This study demonstrates the potential of machine learning techniques to incorporate detailed aerosol microphysics into climate models, improving the fidelity of aerosol–cloud interactions without prohibitive computational costs, thereby advancing climate change projections.

\section{Aerosol classification related works}
\citet{annapurna2024composition} developed and evaluated two aerosol classification schemes based on source and composition using aerosol optical properties from AERONET data collected over Kanpur, India, from 2002 to 2018. Employing five machine learning algorithms—Naïve Bayes, K-Nearest Neighbors, Decision Tree, Support Vector Machine, and Random Forest—the study found that Random Forest and Decision Tree models outperformed others, achieving classification accuracies exceeding 99.5\%. The composition-based classification categorized aerosols into eight types including mixture absorbing, dust (coarse absorbing), and black carbon mixtures, while source-based classification identified urban, maritime, desert, biomass, and arid background aerosols. Seasonal analyses revealed dominance of desert and arid aerosols during pre-monsoon and monsoon seasons and significant presence of black carbon during winter and post-monsoon. This work demonstrates the effectiveness of machine learning methods combined with comprehensive aerosol optical datasets in accurately characterizing aerosol types and sources, providing valuable insights for air quality management and climate impact assessment in heavily polluted regions.

\citet{choi2021first} developed a novel aerosol classification method using a random forest (RF) machine learning approach trained on an AERONET-based aerosol type dataset, integrating satellite measurements from MODIS and TROPOMI. The model classified aerosols into seven types, including pure dust, dust-dominant mixtures, and varying pollution aerosols characterized by absorption strength. The RF-based classification achieved an overall accuracy of 59\%, which improved to 73\% when classes were merged into four categories, demonstrating robust performance across diverse aerosol types with sensitivity to non-spherical particles. Validation against aerosol optical properties such as single scattering albedo (SSA) showed consistent wavelength-dependent behaviors between RF-based and AERONET-based classifications. Compared with earlier threshold-based methods, the RF model exhibited improved classification accuracy and potential for enhancing satellite aerosol retrievals. This study highlights the capability of machine learning to effectively classify aerosols from spaceborne data, supporting improved aerosol monitoring and climate impact assessments.

\citet{christopoulos2018machine} developed a supervised random forest machine learning framework to classify aerosols based on single-particle mass spectrometry (SPMS) data, distinguishing between chemically similar aerosol types with high accuracy. Utilizing a large training dataset of approximately 50,000 spectra representing diverse aerosol categories such as fertile soils, mineral/metallic particles, biological aerosols, and others, the model achieved classification accuracies of about 87\% for specific aerosol types and up to 93\% when aerosols were grouped into broader categories. The approach incorporates dimensionality reduction and chemical feature selection, enabling identification of key mass spectral markers, and demonstrated robust performance on blind test mixtures. This method offers an automated, interpretable tool for detailed aerosol chemical characterization, advancing the understanding of aerosol impacts on climate and atmospheric processes.

\citet{heikkinen2021eight} presented an extensive eight-year characterization of submicron organic aerosol (OA) composition at the SMEAR II station in the boreal forest of Finland using Aerosol Chemical Speciation Monitor (ACSM) data combined with advanced statistical and machine learning techniques. By employing a novel framework that integrated unsupervised feature extraction, positive matrix factorization (PMF) with rolling window approaches, and relaxed chemical mass balance (CMB) modeling, they robustly resolved OA into three subcategories: low-volatility oxygenated OA (LV-OOA), semi-volatile oxygenated OA (SV-OOA), and primary OA (POA). The study found that LV-OOA dominated the OA mass, exhibiting a bimodal seasonal cycle attributed to anthropogenic sources in winter and biogenic influences in summer, while SV-OOA peaked in stable summer nocturnal layers and was influenced by nearby sawmills. POA showed a clear wintertime peak linked to transport from distant sources. Their methodology addressed challenges in long-term remote site data analysis with minimal subjectivity and highlighted the importance of combining machine learning with traditional source apportionment methods to improve aerosol chemical characterization and understanding of seasonal variability in boreal environments.

\citet{song2022understanding} applied positive matrix factorization combined with explainable machine learning (SHAP) to quantify size-resolved aerosol sources and their environmental drivers at the Gruvebadet Observatory in Svalbard, covering particles from 10 nm to 20 \(\mu m\). Nine aerosol factors were identified, including nucleation, biogenic, secondary, anthropogenic, mineral dust, sea salt, and blowing snow aerosols, each dominating specific size ranges and exhibiting distinct seasonal cycles. The study revealed that nucleation and secondary aerosols contribute most to potential cloud condensation nuclei (CCN), with complex nonlinear responses to meteorological parameters such as solar radiation, temperature, boundary layer height, and sea ice exposure. Source region analysis linked aerosol factors to both local Arctic processes and long-range transport from Eurasia, Greenland, and northern North America. This work provides crucial insights into Arctic aerosol composition and dynamics, highlighting the importance of source-specific understanding for improving climate models in rapidly changing polar environments.

\citet{li2022aerosol} applied the unsupervised K-means clustering algorithm to global aerosol simulation data from the EMAC-MADE3 model to identify distinct aerosol regimes across different atmospheric layers. Their approach classified aerosols based on seven primary aerosol properties, including mass concentrations of mineral dust, black carbon, particulate organic matter, sea salt, sulfate-nitrate-ammonium, and particle number concentrations in the Aitken and accumulation modes. The study revealed spatially coherent aerosol clusters that correspond to key emission sources and transport patterns, such as dust over deserts, biomass burning regions, anthropogenic pollution in Asia, and oceanic aerosols. Sensitivity tests on data scaling methods and clustering algorithms confirmed the robustness of their classification. This clustering method facilitates the interpretation of complex aerosol model outputs, aids in evaluating model performance regionally and vertically, and offers a promising tool for enhancing aerosol representation in climate models and satellite retrieval evaluations.

\citet{ningombam2024aerosol} applied a multivariate spectral clustering algorithm to classify global aerosols using five aerosol optical parameters derived from 39,411 daily observations at 150 AERONET sites across six continents from 1993 to 2022. Their analysis identified four primary aerosol types: dust, urban/industrial, biomass burning, and mixed aerosols, with spatial and seasonal variations reflecting regional source influences and atmospheric processes. The study demonstrated that spectral clustering, which projects data into a low-dimensional space based on eigenvalues and eigenvectors of similarity matrices, provides superior classification performance compared to traditional threshold and other clustering methods by effectively capturing complex aerosol optical characteristics without strong shape assumptions. The African and Asian continents showed high dust aerosol contributions, while urban and biomass aerosols dominated Europe, North and South America, and Australia. The findings emphasize the importance of advanced unsupervised machine learning techniques in resolving aerosol heterogeneity for improved remote sensing retrievals and climate impact assessments.

\citet{siomos2020automated} developed an automated aerosol classification technique using ultraviolet (UV) spectral measurements from a double monochromator Brewer spectrophotometer at Thessaloniki, Greece, covering the period 1998–2017. Utilizing a supervised machine learning approach based on the Mahalanobis distance metric, the method classified aerosol mixtures into Fine Non-Absorbing (FNA), Black Carbon (BC), Dust, and Mixed categories with high typing scores when compared to both training datasets and manual classifications. The approach leverages intensive aerosol optical properties such as single scattering albedo (SSA) at 340 nm and extinction Angstrom exponent (EAE) at 320–360 nm, demonstrating its effectiveness despite challenges associated with narrow spectral ranges and measurement uncertainties. Climatological analysis revealed that FNA mixtures dominate the region, with BC and Dust aerosols appearing more frequently after 2007, coinciding with changes in aerosol optical properties likely linked to urban and biomass burning sources. This study highlights the potential of Brewer spectrophotometer data combined with machine learning clustering to provide long-term, detailed aerosol composition information, supporting future climatological and air quality assessments.

\citet{wang2022machine} investigated the complex nonlinear relationships among ozone (O$_3$), secondary organic aerosol (SOA), and volatile organic compounds (VOCs) in Tianjin, China, using two years of continuous measurements and machine learning techniques. Their random forest model identified alkenes (primarily ethylene, propylene, and isoprene) as the most important contributors to O$_3$ formation, while alkanes (C$_n$, \( n \geq 6 \)) and aromatics predominantly influenced SOA formation. The study highlighted the contrasting seasonal patterns of VOCs, O$_3$, and SOA, with VOCs and SOA peaking in colder months and O$_3$ peaking in warmer months. Photochemical consumption analysis revealed that over 80\% of consumed VOCs were alkenes, dominating O$_3$ production, whereas aromatics contributed most to SOA formation, particularly in winter. By combining data-driven machine learning and theoretical calculations, the study provides critical insights into the photochemical mechanisms driving air pollution and offers scientific evidence supporting coordinated seasonal VOC control strategies to effectively reduce both SOA and O$_3$ pollution.

\citet{wang2024linking} combined thermal desorption measurements from FIGAERO-CIMS and machine learning techniques to investigate the volatility and precursor sources of ambient oxygenated organic aerosols (OOA) across diverse urban and marine environments. Their study revealed that OOA, representing approximately 16 ± 13\% of total organic aerosol (OA), primarily consists of secondary and moderate-volatility species with distinct volatility and precursor profiles depending on location, season, and pollution level. Marine OOA was dominated by extremely low volatility organic compounds (ELVOC) and aliphatic species, whereas inland urban OOA was characterized by aromatic species primarily within low- and semi-volatile organic compound ranges. Seasonal and diurnal variations showed lower volatility during summer, pollution days, and daytime relative to winter, clean days, and nighttime. Using a supervised random forest model trained on elemental composition and molecular features, the study successfully assigned precursor types to OOA species, elucidating the complex relationships between precursor VOCs and OOA volatility. The findings provide important insights for understanding secondary organic aerosol formation and offer valuable data to improve atmospheric chemistry models and PM control strategies.

\citet{zhang2025ensemble} proposed an ensemble machine learning approach to improve the long-term source apportionment of atmospheric organic aerosols (OA) from aerosol mass spectrometry (AMS) measurements. Utilizing six years of data from an aerosol chemical speciation monitor (ACSM) in the Paris region, along with OA factor data derived from positive matrix factorization (PMF), the ensemble model combined predictions from random forest (RF), gradient boosting regression trees (GBRT), and extreme gradient boosting (XGBoost) algorithms to enhance prediction accuracy. Compared to individual models, the ensemble approach reduced root-mean-square error (RMSE) by approximately 30\% and increased the correlation coefficient by about 5\%. Sensitivity analyses indicated that training on at least two years of data stabilizes model performance for primary and secondary OA factors, achieving RMSE values of 0.31–0.45 \(\mu g/m^{3}\) and 0.69–0.84 \(\mu g/m^{3}\), respectively. This method offers significant improvements in efficiency for long-term and near-real-time OA source apportionment, with potential applications in air quality monitoring and pollution control policy.

\section{Climate related works}
\citet{chen2022machine} developed a novel machine-learning approach combining meteorological reanalysis and satellite observations to quantify aerosol–cloud interactions (ACI) by isolating aerosol-induced cloud property changes from meteorological variability using the 2014 Holuhraun volcanic eruption as a natural experiment. Their analysis revealed that the primary climate forcing from aerosols arises from a significant increase in cloud fraction (\(\sim 10\%\)), rather than cloud brightening via the Twomey effect, which showed a smaller radiative impact. The study employed a random forest model trained on 20 years of MODIS and ERA5 data to predict cloud properties in the absence of volcanic aerosol perturbations, allowing robust quantification of cloud droplet number concentration, effective radius, liquid water path, and cloud fraction responses to volcanic aerosols. These findings provide critical large-scale observational constraints for climate models, emphasizing the importance of cloud fraction adjustments in aerosol radiative forcing and highlighting the need to improve representations of macro-physical cloud responses in climate projections.

\citet{li2022projected} projected global near-surface aerosol concentrations from 2015 to 2100 using a machine learning approach trained on GEOS-Chem model simulations combined with meteorological fields from CMIP6 multi-model projections and future emission scenarios under various Shared Socioeconomic Pathways (SSPs). The study found that PM$_{2.5}$ concentrations are expected to decrease substantially in East Asia, South Asia, Europe, and North America under low-emission scenarios (SSP1-2.6 and SSP2-4.5), primarily driven by reductions in anthropogenic emissions. However, under high forcing scenarios (SSP5-8.5), climate change alone could increase PM$_{2.5}$ levels by 10–25\% in regions such as northern China and the western United States, highlighting the significant modulation of aerosols by warming climates. The research emphasizes that future climate impacts on aerosol concentrations can be comparable in magnitude to emission-driven changes, underscoring the need to consider both factors in air quality and climate policy planning. Despite uncertainties related to model biases and input data, the study demonstrates the effectiveness of combining machine learning with chemical transport and climate model outputs to improve projections of aerosol evolution under future environmental scenarios.

\citet{redemann2024machine} introduced a novel machine learning (ML) paradigm that utilizes suborbital lidar observations and atmospheric reanalysis data to predict higher-level aerosol properties—specifically aerosol light absorption (ABS) and cloud condensation nuclei (CCN) concentrations—with unprecedented accuracy. Trained on collocated high-accuracy lidar and in situ measurements from multiple airborne campaigns covering diverse aerosol types and environments, the ML models achieved mean relative errors of 21\% for ABS and 13\% for CCN when using suborbital lidar data. The approach was further tested on simulated satellite lidar observations (EarthCARE ATLID), demonstrating promising performance despite higher noise levels inherent in spaceborne measurements. This ML-based retrieval paradigm offers significant advantages over traditional physics-based aerosol retrieval methods by providing vertically resolved aerosol properties near clouds with enhanced spatial and temporal coverage, thus enabling improved constraints on aerosol-cloud interactions and aerosol climate forcing in Earth System Models. The study highlights the potential of combining advanced ML techniques with future spaceborne lidar missions to reduce uncertainties in aerosol climate impacts and enhance climate change projections.

\citet{zhao2022simulating} employed an XGBoost machine learning model trained on extensive seawater DMS measurements and environmental parameters to generate high-resolution monthly DMS emission estimates over the Asian seas for 2017. Coupled with the GEOS-Chem-TOMAS chemical transport model, the study evaluated the contribution of DMS to atmospheric sulfate aerosol and cloud condensation nuclei (CCN) concentrations and quantified its radiative effects regionally. The machine learning-based DMS emissions showed improved spatial and temporal agreement with observations compared to traditional interpolation methods, estimating an annual DMS flux of 1.25 Tg(S), accounting for 15.4\% of anthropogenic sulfur emissions over the domain. The study revealed that DMS-induced sulfate aerosol direct radiative forcing (DRF) ranged from -200 to -20 mW m\(^{-2}\) seasonally, while the cloud-albedo indirect radiative forcing (IRF) was stronger, ranging from -900 to -100 mW m\(^{-2}\), predominantly over remote oceanic areas. These findings highlight the significant climate impact of oceanic DMS emissions in Asia and demonstrate the effectiveness of machine learning methods in improving regional biogenic sulfur emission inventories and associated aerosol–climate interactions.

\section{Cloud related works}
\citet{bender2024machine} applied gradient boosting regression and neural networks to investigate the relative importance of meteorological and aerosol-related parameters in determining cloud microphysical properties across ten geographically diverse regions. Their models demonstrated moderate skill in predicting cloud droplet effective radius (\( R^2 \) up to 0.64) using meteorological variables alone, with only marginal improvements when aerosol optical depth (AOD), sulfur dioxide (SO\(_2\)), and sulfate (SO\(_4\)) data were included. Meteorological factors, especially temperature and geopotential height, were consistently ranked as the most influential predictors, while aerosol variables showed lower and regionally variable importance. Interestingly, the direction of aerosol influence on droplet size was not uniform, with some polluted regions exhibiting counterintuitive positive correlations between AOD and effective radius, likely due to complex aerosol-cloud interactions and regional meteorological conditions. The study highlights the dominant role of meteorology in shaping cloud microphysics and underscores challenges in disentangling aerosol effects from meteorological variability using observational data. These findings provide valuable insights for improving aerosol-cloud interaction representations in climate models.

\citet{fuchs2018building} applied gradient boosting regression trees (GBRT) to analyze satellite, reanalysis, and trajectory data to investigate the drivers of low cloud fraction (CF) and cloud droplet effective radius (REF) in the southeast Atlantic during the biomass-burning season. Their study revealed significant subregional differences in cloud sensitivities: cloud fraction in the southwestern subregion is most influenced by lower tropospheric stability, while surface winds dominate in the northeastern subregion. Aerosol optical depth (AOD) impacts cloud properties primarily in the eastern, more polluted subregions, with free tropospheric temperature playing a key role in cloud droplet size in the northeastern area. Although meteorological factors such as stability, humidity, and air-mass dynamics generally overshadow aerosol effects, aerosols show notable influence on clouds under stable and low-aerosol-loading conditions. This work highlights the importance of considering spatial heterogeneity and nonlinear interactions in aerosol–cloud interactions (ACI) studies and demonstrates the utility of machine learning techniques to disentangle complex environmental influences on low-cloud properties.

\citet{gettelman2021machine} integrated a detailed quasistochastic bin microphysical model of warm rain formation directly into the Community Atmosphere Model version 6 (CAM6), demonstrating that this physically detailed approach significantly alters warm rain processes and climate mean states but at a prohibitive computational cost, increasing runtime by over 400\%. To overcome this, they developed a machine learning-based neural network emulator trained on extensive CAM6-bin model output to efficiently replicate autoconversion and accretion process rates with high accuracy (\(R^2 > 0.98\)) and only an 8\% increase in computational cost compared to the baseline. The emulator captures key improvements in precipitation onset and intensity, cloud liquid water path, and radiative effects, closely reproducing the bin model’s results while enabling long-term climate simulations. This study highlights the potential of combining machine learning with physically detailed microphysics to enhance climate model fidelity and efficiency, offering a promising pathway for improving aerosol–cloud interactions and cloud feedback representations in Earth system models.

\citet{gong2022understanding} presented a comprehensive analysis of 10 years of aerosol microphysical measurements at the Cape Verde Atmospheric Observatory (CVAO), employing an unsupervised K-means clustering algorithm to classify aerosols into five distinct types: marine, freshly formed, mixture, moderate dust, and heavy dust. The study revealed strong seasonal variations, with heavy dust occurring primarily in winter and marine aerosols dominating in spring. Cloud condensation nuclei (CCN) concentrations during dust periods were approximately 2.5 times higher than during marine periods, although particle hygroscopicity (\(\kappa\)) in the size range of 30 to 175 nm showed no significant difference between these aerosol types. Backward trajectory analyses linked aerosol types to their source regions, highlighting the influence of Saharan dust and North Atlantic marine air masses. These long-term data and machine learning classification provide valuable insights into aerosol sources, properties, and their roles in aerosol–cloud interactions, offering a critical baseline for improving atmospheric models and understanding aerosol climate effects in the tropical Atlantic region.

\citet{harder2022physics} developed a physics-informed neural network emulator for the M7 aerosol microphysics module used in the ECHAM-HAM global climate model, achieving an average \( R^{2} \) of 77.1\% in predicting aerosol tendencies while significantly reducing computational costs. The emulator incorporates physical constraints such as mass conservation and positivity through soft regularization and hard constraint layers, ensuring physically consistent predictions suitable for climate model integration. Benchmarks demonstrated a speed-up of over 11,000 times in GPU-only settings and 64 times including data transfer, compared to the original model, highlighting the potential for large-scale climate simulations at finer resolutions and longer timescales. This study represents a significant step towards combining machine learning with physics-based constraints to improve aerosol microphysics representation in Earth system models, offering a promising path to enhanced climate prediction fidelity and efficiency.

\citet{jia2024analysis} developed a near-global, region-specific machine learning framework using Extreme Gradient Boosting (XGB) models to analyze the sensitivity of marine boundary-layer cloud fraction (CLF) to cloud droplet number concentration (Nd) and meteorological factors over nine years of satellite and reanalysis data. Their models explained on average 45\% of the daily CLF variability and revealed a consistently positive Nd–CLF sensitivity, particularly pronounced in stratocumulus-to-cumulus transition regions and the Southern Hemispheric midlatitudes. The study highlighted estimated inversion strength (EIS) and sea surface temperature (SST) as dominant meteorological controls influencing both CLF and the strength of Nd–CLF relationships, with thermodynamic variables generally exerting more influence than dynamical ones. By employing SHapley Additive exPlanations (SHAP) to interpret the models, the work quantified how meteorological conditions modulate aerosol-cloud interactions, providing new insights into cloud fraction adjustments and their potential implications for aerosol radiative forcing under future climate scenarios. The authors also discuss the limitations arising from retrieval uncertainties and the observational nature of the analysis, cautioning that the derived sensitivities represent upper-bound estimates.

\citet{marais2020leveraging} developed a convolutional neural network (CNN)-based methodology that leverages spatial texture features from moderate-resolution multispectral satellite imagery (MODIS and VIIRS) to improve the identification of optically thick aerosols and distinguish among different cloud types, including cumuliform, closed-stratiform, transitional/mixed, and cirrus/high-altitude clouds. By using a carefully curated human-labeled training dataset and pretrained/fine-tuned CNN architectures, their approach outperformed traditional per-pixel spectral and statistical thresholding methods, particularly in accurately detecting aerosol plumes misclassified by standard products. The study demonstrated that patch size significantly affects detection accuracy, with smaller patches better identifying cirrus clouds but larger patches improving aerosol detection sensitivity. Validation against CALIOP lidar data confirmed the improved performance, though limitations remain in distinguishing tenuous aerosols and mixed cloud edges. The work highlights the potential of integrating spatial contextual information via CNNs to enhance global aerosol and cloud classification from satellite imagery, providing valuable insights for atmospheric research and climate studies.

\citet{wang2020machine} developed two Random Forest (RF) machine learning models for cloud mask and thermodynamic-phase detection using spectral observations from the VIIRS instrument aboard Suomi NPP. The models were trained with high-quality reference labels derived from collocated CALIOP lidar measurements over a 4-year period, enabling robust classification of clear sky, liquid water cloud, and ice cloud pixels across seven distinct surface types. The daytime RF model utilized both near-infrared (NIR), shortwave-infrared (SWIR), and thermal infrared (IR) bands, while the all-day model relied on IR bands and surface temperature. Validation against independent CALIOP data in 2017 demonstrated superior or comparable performance relative to existing MODIS and VIIRS cloud mask and phase products, especially over challenging surfaces such as snow and ice. The RF models showed high true positive rates and low false positive rates for both cloud detection and phase classification, with improved consistency across viewing angles and surface types. This study highlights the advantages of ML-based approaches for automated, scalable cloud property retrievals from passive satellite sensors and emphasizes their potential adaptability to future satellite instruments with similar spectral capabilities.

\citet{yorks2021aerosol} presented advanced machine learning techniques applied to data from the Cloud-Aerosol Transport System (CATS) lidar onboard the International Space Station to improve aerosol and cloud detection and classification. Their methods, including wavelet denoising and convolutional neural networks (CNN), enhanced the 1064 nm signal-to-noise ratio (SNR) by 75\%, increased the number of detected atmospheric layers by 30\%, and enabled detection of 40\% more features during daytime at a finer 5 km horizontal resolution compared to the traditional 60 km resolution. The CNN approach improved cloud-aerosol discrimination, particularly in complex scenes and near cloud edges, outperforming the operational algorithms and revealing more homogeneous aerosol layers. While the CNN technique showed limitations at night for detecting optically thin features, the study demonstrates the potential of integrating machine learning with space-based lidar data processing to significantly enhance vertical profiling of aerosols and clouds, thereby supporting improved climate and air quality studies.

\citet{zhao2024studying} utilized nearly four decades of satellite climate data records, reanalysis meteorological data, and machine learning techniques to investigate the aerosol indirect effect (AIE) on deep convective clouds (DCCs) over global oceans across three latitude belts: the northern middle latitude (NML), tropical latitude (TRL), and southern middle latitude (SML). Their analysis using SHapley Additive exPlanation (SHAP) and back-propagation neural network (BPNN) models revealed that aerosol effects manifest most clearly in the NML region, particularly impacting cloud microphysical variables such as cloud particle effective radius (CPER), cloud optical depth (COD), and ice water path (IWP), while macro-physical variables are largely influenced by meteorological covariances. The study identified a sensitive aerosol loading regime, quantified by the aerosol index (AIX), where aerosol effects on cloud properties are most pronounced. Singular value decomposition (SVD) analysis further confirmed that aerosol variability explains a significant fraction of variance in cloud microphysics, with region-specific aerosol transport patterns influencing these interactions. The results highlight the importance of accounting for meteorological feedbacks and covariances when interpreting aerosol-cloud interactions from observations and provide valuable constraints for improving aerosol-cloud parameterizations in climate models.

\section{Mixing state related works}
\citet{hughes2018machine} developed a machine learning framework to predict the global distribution of aerosol mixing state index \(\chi\), which quantifies the degree of internal versus external mixing with respect to aerosol hygroscopicity. Using a large ensemble of particle-resolved aerosol simulations (PartMC-MOSAIC) combined with inputs from a global chemical transport model (GEOS-Chem-TOMAS), they trained gradient-boosted regression trees to estimate \(\chi\) from readily available global model variables. Their results revealed substantial spatial and seasonal variability in aerosol mixing state, with highly internally mixed aerosols (\(\chi\) approaching 100\%) prevalent in polluted regions such as East Asia and the eastern United States, and more externally mixed aerosols (\(\chi\) near 20–30\%) in biomass burning regions like the Amazon basin. This study provides the first global mapping of aerosol mixing state, highlighting regions where simplified assumptions in climate models may lead to significant errors in estimating cloud condensation nuclei concentrations and aerosol radiative effects. The machine learning approach bridges the gap between detailed aerosol microphysics and large-scale models, offering a promising pathway to improve aerosol representation in climate simulations.

\citet{zheng2021estimating} developed machine learning emulators trained on detailed particle-resolved aerosol simulations (PartMC-MOSAIC) to estimate three distinct submicron aerosol mixing state indices—chemical species abundance $\chi_a$, mixing of optically absorbing and nonabsorbing species $\chi_o$, and mixing of hygroscopic and nonhygroscopic species $\chi_h$—from Earth system model outputs (CESM with MAM4). The emulators predicted mixing state indices with an approximate ±10\% error and revealed substantial spatial and seasonal variability globally, with annual averages of 67\%, 68\%, and 56\% for $\chi_a$, $\chi_o$, and $\chi_h$, respectively. The study highlighted that different mixing state indices capture complementary aspects of aerosol composition, with internal and external mixing states varying by region and season. Comparison with limited observational data showed encouraging qualitative agreement, though challenges remain due to differing species definitions and measurement constraints. These results demonstrate that machine learning can effectively bridge detailed aerosol microphysics and large-scale Earth system modeling, providing improved representation of aerosol mixing state crucial for understanding aerosol radiative effects and cloud interactions.

\citet{zheng2021quantifying} quantitatively evaluated the aerosol mixing state representation in the four-mode Modal Aerosol Module (MAM4) of the Community Earth System Model by comparing its predictions of mixing state indices (\(\chi_o\), \(\chi_c\), \(\chi_h\)) against a benchmark machine learning model trained on high-detail particle-resolved simulations. Their results reveal substantial spatial discrepancies, with MAM4 generally overestimating internal mixing of black carbon and primary carbonaceous aerosols at low latitudes and underestimating it at high latitudes. The study identifies key structural biases related to modal definitions and aging parameterizations that limit MAM4’s ability to represent externally mixed particles, particularly in dust-dominated regions. Comparison with available observations suggests that MAM4’s mixing state predictions are less realistic than those from particle-resolved benchmarks, underscoring significant structural uncertainty in current modal model representations of aerosol mixing state. These findings provide critical insights for model developers to improve aerosol process parameterizations, ultimately enhancing aerosol radiative forcing and aerosol–cloud interaction simulations in climate models.

\citet{shen2024improving} integrated a machine learning (ML) model trained on particle-resolved aerosol simulations (PartMC-MOSAIC) into the Community Atmosphere Model version 6 (CAM6) with the Modal Aerosol Module version 4 (MAM4) to improve the representation of black carbon (BC) mixing state. The new MAM4-ML approach partitions accumulation mode particles into BC-containing and BC-free particles using an ML-predicted BC mixing state index \(\chi_{\mathrm{ML}}\), reducing the overestimation of internal mixing present in the default MAM4. This leads to a 19\% reduction in the mixing state index \(\chi_{\mathrm{mode}}\) and a 52\% decrease in the mass ratio of coating to BC (RBC), resulting in a 9\% decrease in BC hygroscopicity and a 20\% reduction in BC activation fraction. Consequently, the BC burden increases globally by 4\%, with notable increases in Arctic surface BC concentrations. The improved mixing state representation better aligns with observations of BC coating thickness and particle composition, especially in remote marine and polar regions. This study demonstrates the potential of coupling ML models with global climate models to enhance aerosol microphysical realism and improve simulations of BC climate impacts and cloud condensation nuclei activity.

\citet{jiang2025integrating} developed a foundation model framework combining pretraining on extensive particle-resolved aerosol simulations (PartMC-MOSAIC) with fine-tuning on limited observational data from the MEGAPOLI campaign to accurately estimate the aerosol mixing state index \(\chi\). By leveraging transfer learning, their ResNet-based model effectively captures complex aerosol mixing state dynamics across diverse environmental conditions, outperforming traditional empirical models. The study highlights that aerosol mass concentrations, particularly organic aerosols, are critical predictors of \(\chi\), while environmental and gas-phase variables contribute complementary information. Performance improves with larger pretraining datasets and sufficient fine-tuning samples, demonstrating the model’s robustness and scalability despite observational data scarcity. This approach offers a practical and computationally efficient pathway to quantify aerosol mixing states, addressing challenges posed by sparse observations and uncertain input data, and provides a promising tool for advancing aerosol research and its implications for climate and health.

\section{Aerosol optical properties related works}
\citet{lary2009machine} applied machine learning techniques, including neural networks and support vector machines (SVM), to correct systematic biases between MODIS aerosol optical depth (AOD) retrievals and ground-based AERONET observations. Their analysis revealed that surface type and viewing geometry significantly influence the bias, with SVMs outperforming neural networks by achieving near-perfect correlations (up to 0.99) with AERONET AOD. The study demonstrated that incorporating ancillary variables such as solar and sensor angles and surface classification enables machine learning models to effectively reduce bias and improve satellite aerosol retrieval accuracy. These findings suggest that machine learning provides a promising approach to refine satellite aerosol products and enhance their utility in climate and air quality research.

\citet{huttunen2016retrieval} evaluated multiple methods for retrieving aerosol optical depth (AOD) from surface solar radiation (SSR) and water vapor content (WVC) measurements, including traditional approaches such as radiative transfer-based look-up tables (LUT) and non-linear regression (NR), as well as machine learning methods like neural networks (NN), random forests (RF), support vector machines (SVM), and Gaussian processes (GP). Using AERONET sun photometer data from Thessaloniki, Greece, they found that machine learning methods, particularly NN and SVM, provided AOD estimates comparable or superior to the LUT approach, effectively capturing variations across low and high AOD ranges. Unlike LUT, which assumes a fixed aerosol single scattering albedo (SSA) and thus exhibits biases dependent on water vapor content, the machine learning methods implicitly accounted for covariations between WVC and SSA, resulting in more accurate retrievals. Their findings suggest that machine learning techniques offer a promising avenue for extending AOD time series beyond satellite and sun photometer records, aiding in better characterization of historical aerosol forcing and enhancing climate studies.

\citet{just2018correcting} developed a novel machine learning approach to correct measurement error in the MAIAC satellite aerosol optical depth (AOD) product over the Northeastern USA by integrating 52 quality control, land use, meteorological, and spatial features. They compared three ensemble algorithms—random forests, gradient boosting, and XGBoost—with XGBoost achieving the best performance, reducing root mean squared prediction error by approximately 43\% relative to raw data. Their corrected AOD product showed improved correlation with ground-level PM$_{2.5}$ measurements by about 10 percentage points, demonstrating enhanced predictive accuracy. The study highlights the importance of addressing measurement error in satellite-derived AOD for improved air pollution modeling and public health applications.

\citet{nabavi2018prediction} compared the performance of multiple machine learning algorithms (MLAs) including Random Forest (RF), Multivariate Adaptive Regression Splines (MARS), Support Vector Machines (SVM), Artificial Neural Networks (ANN), and Multiple Linear Regression (MLR) with deterministic weather prediction models (DMs) such as WRF-chem and MACC in predicting monthly aerosol optical depth (AOD) over West Asia. Using MODIS Deep Blue AOD as the predictand and a comprehensive set of environmental predictors representing dust emission, transport, and deposition, they demonstrated that MLAs, especially SVM and MARS, generally outperformed DMs at spatial and temporal resolutions of 0.25$^\circ$ and monthly scale. Feature selection highlighted the importance of dust source function (SF) and normalized difference vegetation index (NDVI) as key drivers of AOD variability. While both MLAs and DMs underestimated extreme AOD peaks due to dataset limitations and missing predictors, MLAs showed higher agreement with observations and lower prediction errors overall. The study emphasizes the potential of machine learning methods to improve aerosol modeling in regions with sparse observations and complex aerosol life cycles, while also noting the strengths of DMs for short-term forecasting based on physical laws.

\citet{yeom2021estimation} developed a deep neural network (DNN) model to estimate hourly aerosol optical depth (AOD) over Northeast Asia using high temporal and spectral resolution data from the Geostationary Ocean Color Imager (GOCI) satellite. The study compared the DNN with traditional machine learning models, including random forest (RF) and support vector regression (SVR), and physical aerosol retrieval models. The DNN outperformed these approaches, achieving the highest accuracy with a root mean square error (RMSE) of 0.112 and correlation coefficient of 0.863 in hold-out validation, as well as superior spatial and temporal generalization in cross-validation. Despite relying only on multispectral satellite data without ancillary inputs, the DNN effectively captured complex nonlinear relationships between top-of-atmosphere reflectance and ground-based AERONET AOD measurements. The study highlights the promise of deep learning in enhancing aerosol retrieval accuracy and producing reliable hourly AOD products for air quality monitoring and climate research, while noting challenges such as cloud masking and the need for large, diverse training datasets.

\citet{berhane2024comprehensive} conducted a comprehensive spatiotemporal analysis of aerosol optical depth (AOD) and its constituent aerosol species over the Middle East and North Africa (MENA) region from 2003 to 2020 by integrating satellite, reanalysis, and machine learning data. Utilizing the eXtreme Gradient Boosting (XGBoost) model, they achieved high predictive accuracy for AOD across diverse datasets including MODIS, MERRA-2, and CAMSRA, with robust correlations (\( R^{2} \) ranging from 0.76 to 0.96). Their analysis revealed that dust aerosols dominate the region’s aerosol load, especially during spring and summer, driven by frequent dust storms and drought conditions, while black carbon and organic carbon showed increasing trends linked to anthropogenic activities. Feature importance and SHAP analyses identified dust and black carbon as the primary contributors to AOD variability. The study highlighted significant seasonal and regional aerosol patterns influenced by complex interactions between natural emissions, meteorology, and human activities. These insights emphasize the critical role of long-term, high-resolution atmospheric datasets combined with advanced machine learning approaches for improved air quality assessment and climate policy formulation in arid and semi-arid regions.

\citet{luo2018applying} applied a support vector machine (SVM) approach to estimate the integral optical properties of black carbon (BC) fractal aggregates, addressing the computational challenges posed by complex BC morphologies. Trained on numerically exact multiple-sphere T-matrix (MSTM) calculations, the SVM model effectively reproduced extinction, absorption, scattering efficiencies, and asymmetry factors across a wide range of fractal dimensions, monomer numbers, and wavelengths with relative errors within 5\%. The model’s accuracy was maintained even when extended to parameters beyond the initial training set by incorporating additional training data, demonstrating good generalizability. This work provides a promising parameterization framework for BC optical properties that significantly reduces computational cost while preserving accuracy, offering valuable utility for atmospheric modeling, aerosol optical inversions, and climate studies involving BC aggregates with complex morphologies.

\citet{kumar2022correcting} presented a comprehensive evaluation of filter-based aerosol light absorption correction algorithms using co-located measurements from a Particle Soot Absorption Photometer (PSAP) and a Photoacoustic Absorption Spectrometer (PASS) at the Atmospheric Radiation Measurement (ARM) Southern Great Plains (SGP) site. They demonstrated that commonly used analytical correction algorithms, such as those by Virkkula (2010) and Ogren (2010), require site-specific parameter adjustments to achieve reasonable accuracy, yet still exhibit substantial uncertainties. By implementing a random forest regression (RFR) machine learning algorithm incorporating PSAP transmission, uncorrected absorption, nephelometer scattering coefficients, and aerosol chemical composition, they achieved superior correction performance with root mean square errors (RMSE) around 32\% and \(R^2\) values exceeding 0.8 across multiple wavelengths. The RFR model also generalized well to laboratory combustion aerosol datasets, predicting absorption coefficients within 5\% error. This study highlights the potential of data-driven machine learning approaches to overcome limitations of traditional correction methods, providing accurate, scalable corrections for long-term filter-based absorption measurements essential for robust aerosol radiative forcing estimates in climate studies.

\citet{logothetis2023aerosol} demonstrated the feasibility of using sky information from an all-sky imager (ASI), including Red-Green-Blue (RGB) channels and sun saturation area (SAT), combined with supervised machine learning (Light Gradient Boosting Machine, LGBM) to retrieve key aerosol optical properties such as aerosol optical depth (AOD) at 440, 500, and 675 nm, Ångström Exponent (AE) between 440–675 nm, and Fine Mode Fraction (FMF) at 500 nm. Their ML-ASI retrievals showed strong agreement with AERONET reference measurements, with Pearson correlation coefficients \(R\) ranging from 0.89 to 0.95 and moderate errors increasing with higher aerosol loads or coarse particles. Additionally, the retrieved aerosol optical properties enabled aerosol type classification with an overall accuracy of 77.5\%, achieving excellent dust identification exceeding 95\%. Sensitivity analyses highlighted the importance of including total column water vapor (TCWV) and selecting appropriate zenith angle pixel ranges around the sun for optimal retrieval accuracy. This study promotes the use of ASI combined with machine learning as a valuable and complementary tool for continuous aerosol monitoring, particularly in regions lacking dense sun-photometer networks.

\citet{bao2023retrieval} developed a novel random forest (RF) machine learning approach, driven by a differential operator derived from the radiative transfer equation, to retrieve spectral aerosol optical depth (AOD) and Ångström exponent (AE) from Himawari-8 geostationary satellite top-of-atmosphere (TOA) reflectance measurements. The model simultaneously retrieves aerosol properties over both land and ocean with high spatial (5 km) and temporal (hourly) resolution, without reliance on auxiliary coarse-resolution data. Comprehensive validation using AERONET ground-based observations showed strong performance, with \(R^2\) values up to 0.85 for AOD and 0.60 for AE, along with low mean absolute and root mean square errors. The approach effectively captures aerosol spatial and temporal variability, accurately distinguishing aerosol types such as fine-mode pollution and coarse-mode dust. Compared with operational Himawari-8 products and other machine learning models, the RF differential operator model (DORF) demonstrated improved retrieval accuracy and stability, particularly under challenging observation geometries. This study highlights the potential of physically motivated machine learning features to enhance satellite aerosol remote sensing capabilities, offering valuable tools for air quality monitoring and climate research.

\citet{wang2024predictions} characterized 38 typical brown carbon (BrC) chromophores from aerosols collected in Xi’an and applied an interpretable random forest machine learning model combined with SHapley Additive Explanation (SHAP) analysis to explore the relationships between BrC optical properties and chemical composition. Their model achieved high accuracy with Pearson correlation coefficients exceeding 0.93 for the absorption coefficient and above 0.57 for mass absorption efficiency (MAE), explaining over 80\% of variance in absorption and more than 50\% in MAE. The study identified polycyclic aromatic hydrocarbons (PAHs) and oxygenated PAHs (OPAHs) with four and five rings as significant positive contributors to BrC light absorption, suggesting the presence of similar unidentified chromophores influencing BrC optical characteristics. By simplifying BrC optical property estimation using chromophore mass concentrations, this work advances understanding of the chemical-physical linkages in BrC and provides guidance for identifying unknown chromophores, improving atmospheric light absorption modeling and climate impact assessments.

\section{Aerosol size distribution related works}
\citet{ren2020prediction} developed a backpropagation (BP) neural network model to predict aerosol particle size distribution (PSD) using aerosol optical depth (AOD) measurements from a CE-318 sun photometer and kernel functions derived from Mie scattering theory as inputs, with aerodynamic particle sizer (APS 3321) measurements as outputs. By training the model on extensive historical datasets under varying weather conditions—including summer sunny, winter sunny, and dusty days—the BP neural network demonstrated strong predictive capability with a correlation coefficient of 0.99, outperforming radial basis function (RBF) neural networks. The study highlights the feasibility of employing neural networks to invert PSD without directly solving ill-posed Fredholm integral equations, offering an effective and computationally efficient approach to aerosol characterization. This method provides a promising pathway for improving understanding of aerosol direct and indirect effects on climate and air quality through accurate PSD retrievals from remote sensing data.

\citet{barahona2025deep} developed MAMnet, a neural network model designed to predict aerosol size distribution (ASD) and mixing state using bulk aerosol mass and meteorological inputs from single-moment bulk aerosol schemes. Trained on extensive simulations from the Modal Aerosol Module (MAM) within the NASA GEOS system, MAMnet accurately reproduces modal aerosol number concentrations, mass concentrations, and geometric mean diameters across diverse atmospheric conditions, with spatial correlations exceeding 0.9 and mean log-bias generally below 0.1. The model demonstrates strong generalization capabilities when driven by reanalysis data and performs well against ground-based aerosol size observations and global cloud condensation nuclei datasets. Explainable machine learning techniques revealed key drivers such as sulfate, sea salt, dust, temperature, and air density influencing aerosol properties. MAMnet offers a promising approach to improve aerosol microphysics representation in weather forecasting, data assimilation, remote sensing, and climate modeling applications by combining physical accuracy with computational efficiency.

\citet{wu2025estimating} developed deep learning models employing recurrent neural networks (RNNs) to estimate atmospheric aerosol number size distributions (NSDs) using routinely measured trace gas concentrations, meteorological parameters, and total aerosol number concentrations. Trained on 15 years of ambient data from multiple Finnish monitoring stations and tested on an independent coastal site, the models successfully captured temporal variations in particle size ranges from approximately 10 to 500 nm, accurately reproducing total number, surface area, and volume concentrations. The study found that inclusion of total particle number concentration (\(N_{\text{tot}}\)) as an input was critical for accurately estimating smaller particles ($<$75~nm) and new particle formation events. While the models performed well across diverse environments, performance decreased for ultrafine ($<$10~nm) and large particles ($>$500~nm), partly due to instrument limitations and environmental variability not fully represented in training data. Feature importance analysis highlighted \(N_{\text{tot}}\) and carbon monoxide as key predictors. This work demonstrates the potential of advanced machine learning to provide accessible, cost-effective aerosol size distribution estimates, supporting health assessments, air quality management, and climate modeling, especially in regions lacking direct aerosol size measurements.

\section{Others}
\citet{johnson2018using} evaluated the performance of low-cost Shinyei PPD42 aerosol monitors deployed in a dense, heterogeneous urban environment in New York City, comparing them with reference TEOM instruments over a 47-day field campaign. They demonstrated that while linear regression calibration yielded moderate correlations (\(R^2 = 0.36\)–0.53), applying a gradient boosting regression tree (GBRT) model that incorporated meteorological variables significantly improved sensor calibration performance, achieving \(R^2\) values up to 0.76. The GBRT model reduced measurement variability across individual sensors and effectively accounted for environmental influences such as pressure, dew point, and temperature. Despite inherent limitations of low-cost optical sensors, the study highlights the potential for machine learning techniques to enhance the accuracy and reliability of low-cost air quality monitoring networks, enabling finer spatial resolution assessments of particulate matter in complex urban settings. \textcolor{red}{low-cost sensors}

\citet{jin2019machine} addressed the challenge of observation bias in dust storm data assimilation by developing and evaluating two bias correction methods for PM$_{10}$ observations: a conventional chemistry transport model (CTM) and a data-driven machine learning (ML) model based on long short-term memory (LSTM) neural networks. They demonstrated that direct assimilation of raw PM$_{10}$ observations, which include both dust and non-dust aerosols, leads to overestimation and divergence in dust emission inversion. The ML-based bias correction outperformed the CTM in accurately estimating the non-dust fraction of PM$_{10}$, resulting in improved a posteriori dust emission estimates and enhanced dust concentration forecasts. Their variational data assimilation framework incorporating ML bias correction reduced root mean square errors significantly compared to uncorrected assimilation, highlighting the importance of bias correction for reliable dust storm forecasting. This study showcases the effective integration of machine learning into atmospheric data assimilation systems to handle complex, variable observation biases and improve dust emission and concentration estimates. \textcolor{red}{data assimilation}

\citet{shi2022aerosol} developed a deep learning neural network (DLNN) model to predict aerosol iron (Fe) solubility in the global marine atmosphere based on five key variables: particle size range, relative humidity (RH), and molar ratios of sulfate, nitrate, and oxalate to total Fe (TFe) in aerosol particles. Trained on extensive observational data from coastal China and validated against global marine aerosol datasets, the model achieved strong statistical agreement with measured Fe solubility, with Pearson correlation coefficients ranging from 0.73 to 0.97 in most regions except the Atlantic Ocean, where retraining with local data improved predictions. Sensitivity analysis using SHapley Additive exPlanations (SHAP) revealed the oxalate/TFe ratio as the most influential factor controlling Fe solubility, followed by sulfate and nitrate. The study highlights the capability of DLNNs to capture the complex nonlinear interactions governing aerosol Fe solubility and offers a promising approach to improve global biogeochemical modeling of aerosol nutrient inputs to oceans. \textcolor{red}{Aerosol iron solubility}

\citet{song2023lightning} developed a highly accurate machine learning nowcasting model for lightning occurrence over the contiguous United States by integrating aerosol optical depth and composition data with conventional meteorological variables, utilizing high-resolution observations from the Geostationary Lightning Mapper (GLM) onboard GOES-16. The LightGBM model achieved an accuracy of 94.3\%, a probability of detection (POD) of 75\%, and a false alarm ratio (FAR) of 38\%, outperforming traditional baseline models. Importantly, aerosol information significantly enhanced model performance, particularly at higher POD thresholds and in regions with moderate to high aerosol loading, as revealed by interpretable SHapley Additive exPlanations (SHAP) analysis which identified sulfate and organic aerosols as key positive contributors to lightning occurrence, while black carbon exhibited inhibitory effects. The model demonstrated excellent spatial and temporal transferability and provides valuable insights into the complex interactions between aerosols and lightning formation. This study underscores the potential of combining aerosol-informed machine learning with satellite data for improved lightning prediction and understanding aerosol impacts on convective processes. \textcolor{red}{Lightning nowcasting}

\citet{govindasamy2021machine} investigated the application of various machine learning techniques, including Artificial Neural Networks (ANN), Support Vector Regression (SVR), General Regression Neural Network (GRNN), and Random Forest (RF), to predict global solar radiation across South Africa. Uniquely, their study incorporated PM$_{10}$ particulate matter concentrations alongside traditional meteorological variables, demonstrating significant improvement in model accuracy, particularly for ANN models which achieved correlation coefficients up to \( R^{2} = 0.9985 \) and low root mean square errors (RMSE). The inclusion of PM$_{10}$ was found to enhance predictions in models relying on relative humidity, underscoring the importance of aerosols in solar radiation attenuation. Their results emphasize the potential of hybrid machine learning models that combine air quality and meteorological data to provide accurate and computationally efficient solar radiation estimates, which is critical for renewable energy planning and policy-making in regions with limited direct solar measurements. \textcolor{red}{solar radiation prediction}

\section{Reference}
\bibliographystyle{apalike}
\bibliography{main}

\end{document}
